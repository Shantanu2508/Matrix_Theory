\documentclass[journal,12pt,twocolumn]{IEEEtran}

\usepackage{setspace}
\usepackage{gensymb}

\singlespacing


\usepackage[cmex10]{amsmath}

\usepackage{amsthm}

\usepackage{mathrsfs}
\usepackage{txfonts}
\usepackage{stfloats}
\usepackage{bm}
\usepackage{cite}
\usepackage{cases}
\usepackage{subfig}

\usepackage{longtable}
\usepackage{multirow}

\usepackage{enumitem}
\usepackage{mathtools}
%\usepackage{steinmetz}
\usepackage{tikz}
\usepackage{circuitikz}
\usepackage{verbatim}
%\usepackage{tfrupee}
\usepackage[breaklinks=true]{hyperref}

\usepackage{tkz-euclide}

\usetikzlibrary{calc,math}
\usepackage{listings}
    \usepackage{color}                                            %%
    \usepackage{array}                                            %%
    \usepackage{longtable}                                        %%
    \usepackage{calc}                                             %%
    \usepackage{multirow}                                         %%
    \usepackage{hhline}                                           %%
    \usepackage{ifthen}                                           %%
    \usepackage{lscape}     
\usepackage{multicol}
\usepackage{chngcntr}

\DeclareMathOperator*{\Res}{Res}

\renewcommand\thesection{\arabic{section}}
\renewcommand\thesubsection{\thesection.\arabic{subsection}}
\renewcommand\thesubsubsection{\thesubsection.\arabic{subsubsection}}

\renewcommand\thesectiondis{\arabic{section}}
\renewcommand\thesubsectiondis{\thesectiondis.\arabic{subsection}}
\renewcommand\thesubsubsectiondis{\thesubsectiondis.\arabic{subsubsection}}


\hyphenation{op-tical net-works semi-conduc-tor}
\def\inputGnumericTable{}                                 %%

\lstset{
%language=C,
frame=single, 
breaklines=true,
columns=fullflexible
}
\begin{document}


\newtheorem{theorem}{Theorem}[section]
\newtheorem{problem}{Problem}
\newtheorem{proposition}{Proposition}[section]
\newtheorem{lemma}{Lemma}[section]
\newtheorem{corollary}[theorem]{Corollary}
\newtheorem{example}{Example}[section]
\newtheorem{definition}[problem]{Definition}

\newcommand{\BEQA}{\begin{eqnarray}}
\newcommand{\EEQA}{\end{eqnarray}}
\newcommand{\define}{\stackrel{\triangle}{=}}
\bibliographystyle{IEEEtran}
\providecommand{\mbf}{\mathbf}
\providecommand{\pr}[1]{\ensuremath{\Pr\left(#1\right)}}
\providecommand{\qfunc}[1]{\ensuremath{Q\left(#1\right)}}
\providecommand{\sbrak}[1]{\ensuremath{{}\left[#1\right]}}
\providecommand{\lsbrak}[1]{\ensuremath{{}\left[#1\right.}}
\providecommand{\rsbrak}[1]{\ensuremath{{}\left.#1\right]}}
\providecommand{\brak}[1]{\ensuremath{\left(#1\right)}}
\providecommand{\lbrak}[1]{\ensuremath{\left(#1\right.}}
\providecommand{\rbrak}[1]{\ensuremath{\left.#1\right)}}
\providecommand{\cbrak}[1]{\ensuremath{\left\{#1\right\}}}
\providecommand{\lcbrak}[1]{\ensuremath{\left\{#1\right.}}
\providecommand{\rcbrak}[1]{\ensuremath{\left.#1\right\}}}
\theoremstyle{remark}
\newtheorem{rem}{Remark}
\newcommand{\sgn}{\mathop{\mathrm{sgn}}}
\providecommand{\abs}[1]{\left\vert#1\right\vert}
\providecommand{\res}[1]{\Res\displaylimits_{#1}} 
\providecommand{\norm}[1]{\left\lVert#1\right\rVert}
%\providecommand{\norm}[1]{\lVert#1\rVert}
\providecommand{\mtx}[1]{\mathbf{#1}}
\providecommand{\mean}[1]{E\left[ #1 \right]}
\providecommand{\fourier}{\overset{\mathcal{F}}{ \rightleftharpoons}}
%\providecommand{\hilbert}{\overset{\mathcal{H}}{ \rightleftharpoons}}
\providecommand{\system}{\overset{\mathcal{H}}{ \longleftrightarrow}}
	%\newcommand{\solution}[2]{\textbf{Solution:}{#1}}
\newcommand{\solution}{\noindent \textbf{Solution: }}
\newcommand{\cosec}{\,\text{cosec}\,}
\providecommand{\dec}[2]{\ensuremath{\overset{#1}{\underset{#2}{\gtrless}}}}
\newcommand{\myvec}[1]{\ensuremath{\begin{pmatrix}#1\end{pmatrix}}}
\newcommand{\mydet}[1]{\ensuremath{\begin{vmatrix}#1\end{vmatrix}}}
\numberwithin{equation}{subsection}
\makeatletter
\@addtoreset{figure}{problem}
\makeatother
\let\StandardTheFigure\thefigure
\let\vec\mathbf
\renewcommand{\thefigure}{\theproblem}
\def\putbox#1#2#3{\makebox[0in][l]{\makebox[#1][l]{}\raisebox{\baselineskip}[0in][0in]{\raisebox{#2}[0in][0in]{#3}}}}
     \def\rightbox#1{\makebox[0in][r]{#1}}
     \def\centbox#1{\makebox[0in]{#1}}
     \def\topbox#1{\raisebox{-\baselineskip}[0in][0in]{#1}}
     \def\midbox#1{\raisebox{-0.5\baselineskip}[0in][0in]{#1}}
\vspace{3cm}
\title{EE5609 Assignment 2}
\author{SHANTANU YADAV, EE20MTECH12001 }
\maketitle
\newpage
\bigskip
\renewcommand{\thefigure}{\theenumi}
\renewcommand{\thetable}{\theenumi}

The python solution code is available at
\begin{lstlisting}
https://github.com/Shantanu2508/Matrix_Theory/blob/master/Assignment%202/assignment2.py
\end{lstlisting}
%
\section{Problem}
Examine the consistency of the system of the given equations.
\begin{align}
	5x -  y + 4z = 5 \nonumber \\
	2x + 3y + 5z = 2 \\
	5x - 2y + 6z =-1 \nonumber
\end{align}
\section{Solution}
	The given equations can be expressed as
\begin{align}
	\vec{A} \vec{x} = \vec{b}  
\end{align}
where
\begin{align}
	\vec{A} = \myvec{ 5 & -1 & 4 \\ 2 &  3 & 5 \\ 5 & -2 & 6 }
	\quad \text{and} \quad
	\vec{b} = \myvec{ 5 \\  2 \\ -1}
\end{align}
By row reducing the augmented matrix :
\begin{align}
	\myvec{ 5 & -1 & 4 & & 5 \\ 2 & 3 & 5 & & 2 \\ 5 & -2 & 6 & & -1}
\end{align}
\begin{align}
    \xleftrightarrow[]{R_1 \leftarrow \frac{R_1}{5}}
	\myvec{1 & \frac{-1}{5} & \frac{4}{5} & & 1 \\ 
	       2 & 3 & 5 & & 2 \\ 5 & -2 & 6 & & -1}
\end{align}
\begin{align}
	\xleftrightarrow[ R_3 \leftarrow R_3-5R_1]{R_2 \leftarrow R_2-2R_1}
	\myvec{1 & \frac{-1}{5} & \frac{4}{5} & & 1 \\ 
	       0 & \frac{17}{5} & \frac{17}{5} & & 0 \\ 0 & -1 & 2 & & -6}
\end{align}
\begin{align}
	\xleftrightarrow[]{R_2 \leftarrow \frac{5}{17}R_2}
	\myvec{1 & \frac{-1}{5} & \frac{4}{5} & & 1 \\ 
	       0 & 1 & 1 & & 0 \\ 0 & -1 & 2 & & -6}
\end{align}
\begin{align}
	\xleftrightarrow[]{ R_3 \leftarrow R_3+R_2}
        \myvec{1 & \frac{-1}{5} & \frac{4}{5} & & 1 \\ 
	       0 & 1 & 1 & & 0 \\ 0 & 0 & 3 & & -6}
\end{align}
\begin{align}
	\xleftrightarrow[]{R_3 \leftarrow \frac{R_3}{3}}
        \myvec{1 & \frac{-1}{5} & \frac{4}{5} & & 1 \\ 
	       0 & 1 & 1 & & 0 \\ 0 & 0 & 1 & & -2}
\end{align}
\begin{align}
	\xleftrightarrow[]{R_2 \leftarrow R_2-R_3}
        \myvec{1 & \frac{-1}{5} & \frac{4}{5} & & 1 \\ 
	       0 & 1 & 0 & & 2 \\ 0 & 0 & 1 & & -2}
\end{align}
\begin{align}
	\xleftrightarrow[]{R_1 \leftarrow R_1+\frac{1}{5}R_2-\frac{4}{5}R_3}
        \myvec{1 & 0 & 0 & & 3 \\ 
	       0 & 1 & 0 & & 2 \\ 0 & 0 & 1 & & -2}
\end{align}
\begin{align}
	\implies rank \myvec{ 5 & -1 & 4 \\ 2 &  3 & 5 \\ 5 & -2 & 6 }
	&=
	rank \myvec{ 5 & -1 & 4 & & 5 \\ 2 & 3 & 5 & & 2 \\ 5 & -2 & 6 & & -1}
		\nonumber \\
	&= 3 = dim \myvec{ 5 & -1 & 4 \\ 2 &  3 & 5 \\ 5 & -2 & 6 }
\end{align}
i.e., the $rank(\vec{A}) = rank(\vec{A:b}) = 3$. Hence the system of linear equations is consistent, with a unique solution. 

The unique solution is 
\begin{align}
	\vec{x} = \myvec{3\\2\\-2}
\end{align}
\end{document}
