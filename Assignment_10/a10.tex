\documentclass[journal,12pt,twocolumn]{IEEEtran}

\usepackage{setspace}
\usepackage{gensymb}

\singlespacing


\usepackage[cmex10]{amsmath}

\usepackage{amsthm}

\usepackage{mathrsfs}
\usepackage{txfonts}
\usepackage{stfloats}
\usepackage{bm}
\usepackage{cite}
\usepackage{cases}
\usepackage{subfig}

\usepackage{longtable}
\usepackage{multirow}

\usepackage{enumitem}
\usepackage{mathtools}
%\usepackage{steinmetz}
\usepackage{tikz}
\usepackage{circuitikz}
\usepackage{verbatim}
%\usepackage{tfrupee}
\usepackage[breaklinks=true]{hyperref}

\usepackage{tkz-euclide}

\usetikzlibrary{calc,math}
\usepackage{listings}
    \usepackage{color}                                            %%
    \usepackage{array}                                            %%
    \usepackage{longtable}                                        %%
    \usepackage{calc}                                             %%
    \usepackage{multirow}                                         %%
    \usepackage{hhline}                                           %%
    \usepackage{ifthen}                                           %%
    \usepackage{lscape}     
\usepackage{multicol}
\usepackage{chngcntr}

\DeclareMathOperator*{\Res}{Res}

\renewcommand\thesection{\arabic{section}}
\renewcommand\thesubsection{\thesection.\arabic{subsection}}
\renewcommand\thesubsubsection{\thesubsection.\arabic{subsubsection}}

\renewcommand\thesectiondis{\arabic{section}}
\renewcommand\thesubsectiondis{\thesectiondis.\arabic{subsection}}
\renewcommand\thesubsubsectiondis{\thesubsectiondis.\arabic{subsubsection}}


\hyphenation{op-tical net-works semi-conduc-tor}
\def\inputGnumericTable{}                                 %%

\lstset{
%language=C,
frame=single, 
breaklines=true,
columns=fullflexible
}
\begin{document}


\newtheorem{theorem}{Theorem}[section]
\newtheorem{problem}{Problem}
\newtheorem{proposition}{Proposition}[section]
\newtheorem{lemma}{Lemma}[section]
\newtheorem{corollary}[theorem]{Corollary}
\newtheorem{example}{Example}[section]
\newtheorem{definition}[problem]{Definition}

\newcommand{\BEQA}{\begin{eqnarray}}
\newcommand{\EEQA}{\end{eqnarray}}
\newcommand{\define}{\stackrel{\triangle}{=}}
\bibliographystyle{IEEEtran}
\providecommand{\mbf}{\mathbf}
\providecommand{\pr}[1]{\ensuremath{\Pr\left(#1\right)}}
\providecommand{\qfunc}[1]{\ensuremath{Q\left(#1\right)}}
\providecommand{\sbrak}[1]{\ensuremath{{}\left[#1\right]}}
\providecommand{\lsbrak}[1]{\ensuremath{{}\left[#1\right.}}
\providecommand{\rsbrak}[1]{\ensuremath{{}\left.#1\right]}}
\providecommand{\brak}[1]{\ensuremath{\left(#1\right)}}
\providecommand{\lbrak}[1]{\ensuremath{\left(#1\right.}}
\providecommand{\rbrak}[1]{\ensuremath{\left.#1\right)}}
\providecommand{\cbrak}[1]{\ensuremath{\left\{#1\right\}}}
\providecommand{\lcbrak}[1]{\ensuremath{\left\{#1\right.}}
\providecommand{\rcbrak}[1]{\ensuremath{\left.#1\right\}}}
\theoremstyle{remark}
\newtheorem{rem}{Remark}
\newcommand{\sgn}{\mathop{\mathrm{sgn}}}
\providecommand{\abs}[1]{\left\vert#1\right\vert}
\providecommand{\res}[1]{\Res\displaylimits_{#1}} 
\providecommand{\norm}[1]{\left\lVert#1\right\rVert}
%\providecommand{\norm}[1]{\lVert#1\rVert}
\providecommand{\mtx}[1]{\mathbf{#1}}
\providecommand{\mean}[1]{E\left[ #1 \right]}
\providecommand{\fourier}{\overset{\mathcal{F}}{ \rightleftharpoons}}
%\providecommand{\hilbert}{\overset{\mathcal{H}}{ \rightleftharpoons}}
\providecommand{\system}{\overset{\mathcal{H}}{ \longleftrightarrow}}
	%\newcommand{\solution}[2]{\textbf{Solution:}{#1}}
\newcommand{\solution}{\noindent \textbf{Solution: }}
\newcommand{\cosec}{\,\text{cosec}\,}
\providecommand{\dec}[2]{\ensuremath{\overset{#1}{\underset{#2}{\gtrless}}}}
\newcommand{\myvec}[1]{\ensuremath{\begin{pmatrix}#1\end{pmatrix}}}
\newcommand{\mydet}[1]{\ensuremath{\begin{vmatrix}#1\end{vmatrix}}}
\numberwithin{equation}{subsection}
\makeatletter
\@addtoreset{figure}{problem}
\makeatother
\let\StandardTheFigure\thefigure
\let\vec\mathbf
\renewcommand{\thefigure}{\theproblem}
\def\putbox#1#2#3{\makebox[0in][l]{\makebox[#1][l]{}\raisebox{\baselineskip}[0in][0in]{\raisebox{#2}[0in][0in]{#3}}}}
     \def\rightbox#1{\makebox[0in][r]{#1}}
     \def\centbox#1{\makebox[0in]{#1}}
     \def\topbox#1{\raisebox{-\baselineskip}[0in][0in]{#1}}
     \def\midbox#1{\raisebox{-0.5\baselineskip}[0in][0in]{#1}}
\vspace{3cm}
\title{EE5609 Assignment 10}
\author{SHANTANU YADAV, EE20MTECH12001 }
\maketitle
\newpage
\bigskip
\renewcommand{\thefigure}{\theenumi}
\renewcommand{\thetable}{\theenumi}
\section{Problem}
If $\vec{F}$ is a field, verify that vector space of all ordered n-tuples $\vec{F}^n$ is a 
vector space over the field $\vec{F}$.
\section{Solution}
Let $\vec{F}^n$ be a set of all ordered n-tuples over $\vec{F}$ i.e
\begin{align}
	\vec{F}^n= \cbrak{\myvec{\alpha_1 \\ \alpha_2 \\ \vdots \\ \alpha_n} : 
	\alpha_1, \alpha_2, \ldots, \alpha_n \in \vec{F} }
\end{align}
For $\vec{F}^n$ to be a vector space over $\vec{F}$ it must satisfy the 
closure property of vector addition and scalar multiplication. \\
\\
{\bf Vector Addition in $\vec{F}^n$ :} \\
Let $\vec{\alpha} = \myvec{\alpha_1\\ \alpha_2\\ \vdots\\ \alpha_n}$ and
$\vec{\beta} = \myvec{\beta_1\\ \beta_2\\ \vdots\\ \beta_n} \in \vec{F}^n$ 
then 
\begin{align}
	\vec{\alpha} + \vec{\beta} 
	&=\myvec{\alpha_1\\ \alpha_2\\ \vdots\\ \alpha_n}
	+\myvec{\beta_1\\ \beta_2\\ \vdots\\ \beta_n} \\
	&=\myvec{\alpha_1+\beta_1\\ \alpha_2+\beta_2\\ \vdots\\ \alpha_n+\beta_n} 
\end{align}
Since 
\begin{align}
	\alpha_i+\beta_i \in \vec{F} \ \forall \  i=1,2,\cdots,n \\
	\implies \vec{\alpha}+\vec{\beta} \in \vec{F}^n
\end{align}

{\bf Scalar multiplication in $\vec{F}^n$ over $\vec{F}$ :} \\
Let $\vec{\alpha} = \myvec{\alpha_1\\ \alpha_2\\ \vdots\\ \alpha_n} 
	\in \vec{F}^n$ and  $ a \in \vec{F}$ then
\begin{align}
	a\vec{\alpha}=\myvec{a\alpha_1\\ a\alpha_2\\ \vdots\\ a\alpha_n}
\end{align}
Since
\begin{align}
	a\alpha_i \in \vec{F} \ \forall \ i=1,2\cdots,n \\
	\implies a\vec{\alpha} \in \vec{F}^n
\end{align}
\\
{\bf Associativity of addition in $\vec{F}^n$ :} \\
Let $\vec{\alpha}=\myvec{\alpha_1\\ \alpha_2\\ \vdots\\ \alpha_n}, \ 
\vec{\beta}=\myvec{\beta_1\\ \beta_2\\ \vdots\\ \beta_n}, 
\ \vec{\gamma}=\myvec{\gamma_1\\ \gamma_2\\ \vdots\\ \gamma_n} 
\in \vec{F}^n$ then
\begin{align}
	\vec{\alpha}+(\vec{\beta}+\vec{\gamma}) &= 
	\myvec{\alpha_1\\ \alpha_2\\ \vdots\\ \alpha_n} + 
      \myvec{\beta_1+\gamma_1\\ \beta_2+\gamma_2\\ \vdots\\ \beta_n+\gamma_n} \\
	&= \myvec{\alpha_1+\beta_1+\gamma_1\\ \alpha_2+\beta_2+\gamma_2\\
	\vdots\\ \alpha_n+\beta_n+\gamma_n} \\
	&= \myvec{\alpha_1+\beta_1\\ \alpha_2+\beta_2\\ \vdots\\
	\alpha_n+\beta_n} + \myvec{\gamma_1\\ \gamma_2\\ \vdots\\ \gamma_n} \\
	&=(\alpha+\beta)+\gamma
\end{align}

{\bf Existence of additive identity in $\vec{F}^n$ :} \\ 
We have $\myvec{0\\ 0\\ \vdots\\ 0} \in \vec{F}^n$
and $\alpha=\myvec{\alpha_1\\ \alpha_2\\ \vdots\\ \alpha_n} \in \vec{F}^n$ then
\begin{align}
	\myvec{\alpha_1\\ \alpha_2\\ \vdots\\ \alpha_n}
	+\myvec{0\\ 0\\ \vdots\\ 0}
	&=\myvec{\alpha_1+0\\ \alpha_2+0\\ \vdots\\ \alpha_n+0} \\
	&=\myvec{\alpha_1\\ \alpha_2\\ \vdots\\ \alpha_n} 
\end{align}
Therefore $\myvec{0\\ 0\\ \vdots\\ 0}$ is the additive identity in 
$\vec{F}^n$.\\ \\
{\bf Existence of additive inverse of each element of $\vec{F}^n$ :} \\
If $\myvec{\alpha_1\\ \alpha_2\\ \vdots\\ \alpha_n} \in \vec{F}^n$ then 
$\myvec{-\alpha_1\\ -\alpha_2\\ \vdots\\ -\alpha_n} \in \vec{F}^n$. 
Also we have
\begin{align}
	\myvec{-\alpha_1\\ -\alpha_2\\ \cdots\\ -\alpha_n}
	+\myvec{\alpha_1\\ \alpha_2\\ \cdots\\ \alpha_n}
	&=\myvec{0\\ 0\\ \vdots\\ 0}
\end{align}
Therefore $\myvec{-\alpha_1\\ -\alpha_2\\ \vdots\\ -\alpha_n}$ is the 
additive inverse of $\myvec{\alpha_1\\ \alpha_2\\ \vdots\\ \alpha_n}$.
Thus $\vec{F}^n$ is an abelian group with respect to addition. \\

Futher we observe that
\begin{enumerate}
\item If $a$ $\in \vec{F}$ and 
	$\vec{\alpha} = \myvec{\alpha_1\\ \alpha_2\\ \vdots\\ \alpha_n}, \
	\vec{\beta}=\myvec{\beta_1\\ \beta_2\\ \vdots\\ \beta_n} \in \vec{F}^n$ 
then
\begin{align}
	a(\vec{\alpha}+\vec{\beta}) &=
	a\myvec{\alpha_1+\beta_1\\ \alpha_2+\beta_2\\ \vdots\\ 
			\alpha_n+\beta_n} 
\end{align}
\begin{align}
	&=\myvec{a[\alpha_1+\beta_1]\\ a[\alpha_2+\beta_2]\\ \vdots\\
			a[\alpha_n+\beta_n]} \\
	&=\myvec{a\alpha_1+a\beta_1\\ a\alpha_2+a\beta_2\\ \vdots\\ 
			a\alpha_n+a\beta_n} \\
	&\myvec{a\alpha_1\\ a\alpha_2\\ \vdots\\ a\alpha_n}
	+ \myvec{a\beta_1\\ a\beta_2\\ \vdots\\ a\beta_n} \\
	&=a\myvec{\alpha_1\\ \alpha_2\\ \vdots\\ \alpha_n}
	+ a\myvec{\beta_1\\ \beta_2\\ \vdots\\ \beta_n} \\
	&=a\alpha+a\beta
\end{align}
\item If $a$,$b$ $\in \vec{F}$ and 
	$\myvec{\alpha_1\\ \alpha_2\\ \vdots\\ \alpha_n} \in \vec{F}^n$ then
\begin{align}
	(a+b)\alpha
	&=\myvec{[a+b]\alpha_1\\ [a+b]\alpha_2\\ \vdots\\ [a+b]\alpha_n} 
\end{align}
\begin{align}
	&=\myvec{a\alpha_1+b\alpha_1\\ a\alpha_2+b\alpha_2\\ \vdots\\
		a\alpha_n+b\alpha_n} \\ 
	&=\myvec{a\alpha_1\\ a\alpha_2\\ \vdots\\ a\alpha_n}+
		\myvec{b\alpha_1\\ b\alpha_2\\ \cdots,b\alpha_n} \\
	&=a\myvec{\alpha_1\\ \alpha_2\\ \vdots\\ \alpha_n}+
		b\myvec{\alpha_1\\ \alpha_2\\ \vdots\\ \alpha_n} \\
	&=a\alpha+b\alpha
\end{align}
\item If $a$,$b$ $\in \vec{F}$ and 
	$\myvec{\alpha_1\\ \alpha_2\\ \vdots\\ \alpha_n} \in \vec{F}^n$ 
then
\begin{align}
     (ab)\alpha&=\myvec{[ab]\alpha_1\\ [ab]\alpha_2\\ \vdots\\ [ab]\alpha_n} \\
	  &=\myvec{a[b\alpha_1]\\ a[b\alpha_2]\\ \vdots\\ a[b\alpha_n]} \\
	  &=a\myvec{b\alpha_1\\ b\alpha_2\\ \vdots\\ b\alpha_n} \\
	  &=a(b\alpha)
\end{align}
\item If 1 is the unity element of $\vec{F}$ and 
	$\alpha=\myvec{\alpha_1\\ \alpha_2\\ \alpha_n} \in \vec{F}^n$ then
\begin{align}
	1\alpha &=\myvec{1\alpha_1\\ 1\alpha_2\\ \vdots\\ 1\alpha_n} \\
		&=\myvec{\alpha_1\\ \alpha_2\\ \vdots\\ \alpha_n} \\
		&= \alpha
\end{align}
\end{enumerate}
Hence $\vec{F}^n$ is a vector space over $\vec{F}$.
\end{document}
