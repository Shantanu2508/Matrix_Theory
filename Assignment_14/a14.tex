\documentclass[journal,12pt,twocolumn]{IEEEtran}

\usepackage{setspace}
\usepackage{gensymb}

\singlespacing


\usepackage[cmex10]{amsmath}

\usepackage{amsthm}

\usepackage{mathrsfs}
\usepackage{txfonts}
\usepackage{stfloats}
\usepackage{bm}
\usepackage{cite}
\usepackage{cases}
\usepackage{subfig}

\usepackage{longtable}
\usepackage{multirow}

\usepackage{enumitem}
\usepackage{mathtools}
%\usepackage{steinmetz}
\usepackage{tikz}
\usepackage{circuitikz}
\usepackage{verbatim}
%\usepackage{tfrupee}
\usepackage[breaklinks=true]{hyperref}

\usepackage{tkz-euclide}

\usetikzlibrary{calc,math}
\usepackage{listings}
    \usepackage{color}                                            %%
    \usepackage{array}                                            %%
    \usepackage{longtable}                                        %%
    \usepackage{calc}                                             %%
    \usepackage{multirow}                                         %%
    \usepackage{hhline}                                           %%
    \usepackage{ifthen}                                           %%
    \usepackage{lscape}     
\usepackage{multicol}
\usepackage{chngcntr}
%\usepackage{enumerate,float}
\DeclareMathOperator*{\Res}{Res}

\renewcommand\thesection{\arabic{section}}
\renewcommand\thesubsection{\thesection.\arabic{subsection}}
\renewcommand\thesubsubsection{\thesubsection.\arabic{subsubsection}}

\renewcommand\thesectiondis{\arabic{section}}
\renewcommand\thesubsectiondis{\thesectiondis.\arabic{subsection}}
\renewcommand\thesubsubsectiondis{\thesubsectiondis.\arabic{subsubsection}}


\hyphenation{op-tical net-works semi-conduc-tor}
\def\inputGnumericTable{}                                 %%

\lstset{
%language=C,
frame=single, 
breaklines=true,
columns=fullflexible
}
\begin{document}


\newtheorem{theorem}{Theorem}[section]
\newtheorem{problem}{Problem}
\newtheorem{proposition}{Proposition}[section]
\newtheorem{lemma}{Lemma}[section]
\newtheorem{corollary}[theorem]{Corollary}
\newtheorem{example}{Example}[section]
\newtheorem{definition}[problem]{Definition}

\newcommand{\BEQA}{\begin{eqnarray}}
\newcommand{\EEQA}{\end{eqnarray}}
\newcommand{\define}{\stackrel{\triangle}{=}}
\bibliographystyle{IEEEtran}
\providecommand{\mbf}{\mathbf}
\providecommand{\pr}[1]{\ensuremath{\Pr\left(#1\right)}}
\providecommand{\qfunc}[1]{\ensuremath{Q\left(#1\right)}}
\providecommand{\sbrak}[1]{\ensuremath{{}\left[#1\right]}}
\providecommand{\lsbrak}[1]{\ensuremath{{}\left[#1\right.}}
\providecommand{\rsbrak}[1]{\ensuremath{{}\left.#1\right]}}
\providecommand{\brak}[1]{\ensuremath{\left(#1\right)}}
\providecommand{\lbrak}[1]{\ensuremath{\left(#1\right.}}
\providecommand{\rbrak}[1]{\ensuremath{\left.#1\right)}}
\providecommand{\cbrak}[1]{\ensuremath{\left\{#1\right\}}}
\providecommand{\lcbrak}[1]{\ensuremath{\left\{#1\right.}}
\providecommand{\rcbrak}[1]{\ensuremath{\left.#1\right\}}}
\theoremstyle{remark}
\newtheorem{rem}{Remark}
\newcommand{\sgn}{\mathop{\mathrm{sgn}}}
\providecommand{\abs}[1]{\left\vert#1\right\vert}
\providecommand{\res}[1]{\Res\displaylimits_{#1}} 
\providecommand{\norm}[1]{\left\lVert#1\right\rVert}
%\providecommand{\norm}[1]{\lVert#1\rVert}
\providecommand{\mtx}[1]{\mathbf{#1}}
\providecommand{\mean}[1]{E\left[ #1 \right]}
\providecommand{\fourier}{\overset{\mathcal{F}}{ \rightleftharpoons}}
%\providecommand{\hilbert}{\overset{\mathcal{H}}{ \rightleftharpoons}}
\providecommand{\system}{\overset{\mathcal{H}}{ \longleftrightarrow}}
	%\newcommand{\solution}[2]{\textbf{Solution:}{#1}}
\newcommand{\solution}{\noindent \textbf{Solution: }}
\newcommand{\cosec}{\,\text{cosec}\,}
\providecommand{\dec}[2]{\ensuremath{\overset{#1}{\underset{#2}{\gtrless}}}}
\newcommand{\myvec}[1]{\ensuremath{\begin{pmatrix}#1\end{pmatrix}}}
\newcommand{\mydet}[1]{\ensuremath{\begin{vmatrix}#1\end{vmatrix}}}
\numberwithin{equation}{subsection}
\makeatletter
\@addtoreset{figure}{problem}
\makeatother
\let\StandardTheFigure\thefigure
\let\vec\mathbf
\renewcommand{\thefigure}{\theproblem}
\def\putbox#1#2#3{\makebox[0in][l]{\makebox[#1][l]{}\raisebox{\baselineskip}[0in][0in]{\raisebox{#2}[0in][0in]{#3}}}}
     \def\rightbox#1{\makebox[0in][r]{#1}}
     \def\centbox#1{\makebox[0in]{#1}}
     \def\topbox#1{\raisebox{-\baselineskip}[0in][0in]{#1}}
     \def\midbox#1{\raisebox{-0.5\baselineskip}[0in][0in]{#1}}
\vspace{3cm}
\title{EE5609 Assignment 14}
\author{SHANTANU YADAV, EE20MTECH12001 }
\maketitle
\newpage
\bigskip
\renewcommand{\thefigure}{\theenumi}
\renewcommand{\thetable}{\theenumi}

\section{Problem}
Let $\vec{T}$ be a linear operator on $\vec{R}^3$ defined by
\begin{align*}
	\vec{T}\myvec{x_1\\x_2\\x_3}=\myvec{3x_1\\x_1-x_2\\2x_1+x_2+x_3}
\end{align*}
Is $\vec{T}$ invertible? If so, find a rule for $\vec{T}^{-1}$ like the one which defines T.
\section{Explanation}
The transformed vector can be re-written by expanding the columns as follows
\begin{align}
	\myvec{3x_1\\x_1-x_2\\2x_1+x_2+x_3}
	&=\myvec{3\\1\\2}x_1+\myvec{0\\-1\\1}x_2+\myvec{0\\0\\1}x_3\\
	&=\myvec{3&0&0\\
		 1&-1&0\\
		 2&1&1}\myvec{x_1\\x_2\\x_3} \\
	&\implies \vec{T}=\myvec{3&0&0\\
                 1&-1&0\\
                 2&1&1}
\end{align}
Using Gauss-Jordan Elimination to find the inverse of $\vec{T}$, if it exists
\begin{align}
	\myvec{3&0&0&|&1&0&0\\
	       1&-1&0&|&0&1&0\\
	       2&1&1&|&0&0&1}\\
	\xleftrightarrow[]{R_1\leftarrow\frac{R_1}{3}}
	\myvec{1&0&0&|&\frac{1}{3}&0&0\\
               1&-1&0&|&0&1&0\\
               2&1&1&|&0&0&1}\\
	\xleftrightarrow[R_3\leftarrow R_3-2R_1]{R_2\leftarrow R_2-R_1}
	\myvec{1&0&0&|&\frac{1}{3}&0&0\\
	       0&-1&0&|&-\frac{1}{3}&1&0\\
	       0&1&1&|&-\frac{2}{3}&0&1}\\
        \xleftrightarrow[]{R_2\leftarrow \ -R_2}
	\myvec{1&0&0&|&\frac{1}{3}&0&0\\
               0&1&0&|&\frac{1}{3}&-1&0\\
               0&1&1&|&-\frac{2}{3}&0&1}\\
        \xleftrightarrow[]{R_3\leftarrow R_3-R_2}
	\myvec{1&0&0&|&\frac{1}{3}&0&0\\
               0&1&0&|&\frac{1}{3}&-1&0\\
               0&0&1&|&-1&1&1}
\end{align}
Since $rank(\vec{T})=3, \ \vec{T}$ is invertible and the inverse is
\begin{align}
	\vec{T}^{-1}=\myvec{\frac{1}{3}&0&0\\ 
                            \frac{1}{3}&-1&0\\
                            -1&1&1}
\end{align}
Now consider any vector $\myvec{x_1\\x_2\\x_3} \in \vec{R}^3$, then
\begin{align}
	\vec{T}^{-1}\myvec{x_1\\x_2\\x_3}&=\myvec{\frac{1}{3}&0&0\\
                            \frac{1}{3}&-1&0\\
                            -1&1&1}\myvec{x_1\\x_2\\x_3}\\
			    &=\myvec{\frac{x_1}{3}\\\frac{x_1}{3}-x_2\\-x_1+x_2+x_3}
\end{align}
Therefore the transformation $\vec{T}^{-1}$ is defined on $\vec{R}^3 $ as
\begin{align}
\vec{T}^{-1}\myvec{x_1\\x_2\\x_3}=\myvec{\frac{x_1}{3}\\\frac{x_1}{3}-x_2\\-x_1+x_2+x_3}
\end{align}
\end{document}
