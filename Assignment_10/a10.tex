\documentclass[journal,12pt,twocolumn]{IEEEtran}

\usepackage{setspace}
\usepackage{gensymb}

\singlespacing


\usepackage[cmex10]{amsmath}

\usepackage{amsthm}

\usepackage{mathrsfs}
\usepackage{txfonts}
\usepackage{stfloats}
\usepackage{bm}
\usepackage{cite}
\usepackage{cases}
\usepackage{subfig}

\usepackage{longtable}
\usepackage{multirow}

\usepackage{enumitem}
\usepackage{mathtools}
%\usepackage{steinmetz}
\usepackage{tikz}
\usepackage{circuitikz}
\usepackage{verbatim}
%\usepackage{tfrupee}
\usepackage[breaklinks=true]{hyperref}

\usepackage{tkz-euclide}

\usetikzlibrary{calc,math}
\usepackage{listings}
    \usepackage{color}                                            %%
    \usepackage{array}                                            %%
    \usepackage{longtable}                                        %%
    \usepackage{calc}                                             %%
    \usepackage{multirow}                                         %%
    \usepackage{hhline}                                           %%
    \usepackage{ifthen}                                           %%
    \usepackage{lscape}     
\usepackage{multicol}
\usepackage{chngcntr}

\DeclareMathOperator*{\Res}{Res}

\renewcommand\thesection{\arabic{section}}
\renewcommand\thesubsection{\thesection.\arabic{subsection}}
\renewcommand\thesubsubsection{\thesubsection.\arabic{subsubsection}}

\renewcommand\thesectiondis{\arabic{section}}
\renewcommand\thesubsectiondis{\thesectiondis.\arabic{subsection}}
\renewcommand\thesubsubsectiondis{\thesubsectiondis.\arabic{subsubsection}}


\hyphenation{op-tical net-works semi-conduc-tor}
\def\inputGnumericTable{}                                 %%

\lstset{
%language=C,
frame=single, 
breaklines=true,
columns=fullflexible
}
\begin{document}


\newtheorem{theorem}{Theorem}[section]
\newtheorem{problem}{Problem}
\newtheorem{proposition}{Proposition}[section]
\newtheorem{lemma}{Lemma}[section]
\newtheorem{corollary}[theorem]{Corollary}
\newtheorem{example}{Example}[section]
\newtheorem{definition}[problem]{Definition}

\newcommand{\BEQA}{\begin{eqnarray}}
\newcommand{\EEQA}{\end{eqnarray}}
\newcommand{\define}{\stackrel{\triangle}{=}}
\bibliographystyle{IEEEtran}
\providecommand{\mbf}{\mathbf}
\providecommand{\pr}[1]{\ensuremath{\Pr\left(#1\right)}}
\providecommand{\qfunc}[1]{\ensuremath{Q\left(#1\right)}}
\providecommand{\sbrak}[1]{\ensuremath{{}\left[#1\right]}}
\providecommand{\lsbrak}[1]{\ensuremath{{}\left[#1\right.}}
\providecommand{\rsbrak}[1]{\ensuremath{{}\left.#1\right]}}
\providecommand{\brak}[1]{\ensuremath{\left(#1\right)}}
\providecommand{\lbrak}[1]{\ensuremath{\left(#1\right.}}
\providecommand{\rbrak}[1]{\ensuremath{\left.#1\right)}}
\providecommand{\cbrak}[1]{\ensuremath{\left\{#1\right\}}}
\providecommand{\lcbrak}[1]{\ensuremath{\left\{#1\right.}}
\providecommand{\rcbrak}[1]{\ensuremath{\left.#1\right\}}}
\theoremstyle{remark}
\newtheorem{rem}{Remark}
\newcommand{\sgn}{\mathop{\mathrm{sgn}}}
\providecommand{\abs}[1]{\left\vert#1\right\vert}
\providecommand{\res}[1]{\Res\displaylimits_{#1}} 
\providecommand{\norm}[1]{\left\lVert#1\right\rVert}
%\providecommand{\norm}[1]{\lVert#1\rVert}
\providecommand{\mtx}[1]{\mathbf{#1}}
\providecommand{\mean}[1]{E\left[ #1 \right]}
\providecommand{\fourier}{\overset{\mathcal{F}}{ \rightleftharpoons}}
%\providecommand{\hilbert}{\overset{\mathcal{H}}{ \rightleftharpoons}}
\providecommand{\system}{\overset{\mathcal{H}}{ \longleftrightarrow}}
	%\newcommand{\solution}[2]{\textbf{Solution:}{#1}}
\newcommand{\solution}{\noindent \textbf{Solution: }}
\newcommand{\cosec}{\,\text{cosec}\,}
\providecommand{\dec}[2]{\ensuremath{\overset{#1}{\underset{#2}{\gtrless}}}}
\newcommand{\myvec}[1]{\ensuremath{\begin{pmatrix}#1\end{pmatrix}}}
\newcommand{\mydet}[1]{\ensuremath{\begin{vmatrix}#1\end{vmatrix}}}
\numberwithin{equation}{subsection}
\makeatletter
\@addtoreset{figure}{problem}
\makeatother
\let\StandardTheFigure\thefigure
\let\vec\mathbf
\renewcommand{\thefigure}{\theproblem}
\def\putbox#1#2#3{\makebox[0in][l]{\makebox[#1][l]{}\raisebox{\baselineskip}[0in][0in]{\raisebox{#2}[0in][0in]{#3}}}}
     \def\rightbox#1{\makebox[0in][r]{#1}}
     \def\centbox#1{\makebox[0in]{#1}}
     \def\topbox#1{\raisebox{-\baselineskip}[0in][0in]{#1}}
     \def\midbox#1{\raisebox{-0.5\baselineskip}[0in][0in]{#1}}
\vspace{3cm}
\title{EE5609 Assignment 10}
\author{SHANTANU YADAV, EE20MTECH12001 }
\maketitle
\newpage
\bigskip
\renewcommand{\thefigure}{\theenumi}
\renewcommand{\thetable}{\theenumi}
\section{Problem}
If $\mathbb{F}$ is a field, verify that vector space of all ordered n-tuples $\mathbb{F}^n$ is a 
vector space over the field $\mathbb{F}$.
\section{Solution}
Let $\mathbb{F}^n$ be a set of all ordered n-tuples over $\mathbb{F}$ i.e
\begin{align}
	\mathbb{F}^n= \{(a_1,a_2,\cdots,a_n) : a_1,a_2,\cdots,a_n \in \mathbb{F} \}
\end{align}
For $\mathbb{F}^n$ to be a vector space over $\mathbb{F}$ it must satisfy the closure property 
of vector addition and scalar multiplication. \\
{\bf Vector Addition in $\mathbb{F}^n$ :} \\
Let $\alpha = (\alpha_1,\alpha_2,\cdots,\alpha_n)$ and
$\beta = (\beta_1,\beta_2,\cdots,\beta_n) \in \mathbb{F}^n$ then 
\begin{align}
	\alpha + \beta &= (\alpha_1,\alpha_2,\cdots,\alpha_n) + (\beta_1,\beta_2,\cdots,\beta_n) \\
		       &= (\alpha_1+\beta_1,\alpha_2+\beta_2,\cdots,\alpha_n+\beta_n) 
\end{align}
Since 
\begin{align}
	\alpha_i+\beta_i \in \mathbb{F} \ \forall \  i=1,2,\cdots,n \\
	\implies \alpha+\beta \in \mathbb{F}^n
\end{align}
{\bf Scalar multiplication in $\mathbb{F}^n$ over $\mathbb{F}$ :} \\
Let $\alpha = (\alpha_1,\alpha_2,\cdots,\alpha_n) \in \mathbb{F}^n$ and a $\in \mathbb{F}$ then
\begin{align}
	a\alpha=(a\alpha_1,a\alpha_2,\cdots,a\alpha_n)
\end{align}
Since
\begin{align}
	a\alpha_i \in \mathbb{F} \ \forall \ i=1,2\cdots,n \\
	\implies a\alpha \in \mathbb{F}^n
\end{align}
{\bf Associativity of addition in $\mathbb{F}^n$ :} \\
Let $\alpha=(\alpha_1,\alpha_2,\cdots,\alpha_n), \ \beta=(\beta_1,\beta_2,\cdots,\beta_n), 
\ \gamma=(\gamma_1,\gamma_2,\cdots,\gamma_n) \in \mathbb{F}^n$ then
\begin{align}
	\alpha+(\beta+\gamma) &= (\alpha_1,\alpha_2,\cdots,\alpha_n) + (\beta_1+\gamma_1,\beta_2+\gamma_2,\cdots,\beta_n+\gamma_n) \\
	&= (\alpha_1+\beta_1+\gamma_1,\alpha_2+\beta_2+\gamma_2,\cdots,\alpha_n+\beta_n+\gamma_n) \\
	&= (\alpha_1+\beta_1,\alpha_2+\beta_2,\cdots,\alpha_n+\beta_n) + (\gamma_1,\gamma_2,\cdots,\gamma_n) \\
	&=(\alpha+\beta)+\gamma
\end{align}
{\bf Existence of additive identity in $\mathbb{F}^n$ :} \\
We have $(0,0,\cdots,0) \in \mathbb{F}^n$
and $\alpha=(\alpha_1,\alpha_2,\cdots,\alpha_n) \in \mathbb{F}^n$ then
\begin{align}
	(\alpha_1,\alpha_2,\cdots,\alpha_n)+(0,0,\cdots,0)&=(\alpha_1+0,\alpha_2+0,\cdots,\alpha_n+0) \\
	&=(\alpha_1,\alpha_2,\cdots,\alpha_n) 
\end{align}
Therefore $(0,0,\cdots,0)$ is the additive identity in $\mathbb{F}^n$.\\
{\bf Existence of additive inverse of each element of $\mathbb{F}^n$ :} \\
If $(\alpha_1,\alpha_2,\cdots,\alpha_n) \in \mathbb{F}^n$ then $(-\alpha_1,-\alpha_2,\cdots,-\alpha_n)
\in \mathbb{F}^n$. Also we have
\begin{align}
	(-\alpha_1,-\alpha_2,\cdots,-\alpha_n)+(\alpha_1,\alpha_2,\cdots,\alpha_n)&=(0,0,\cdots,0)
\end{align}
Therefore $(-\alpha_1,-\alpha_2,\cdots,-\alpha_n)$ is the additive inverse of $(\alpha_1,\alpha_2,\cdots,\alpha_n)$.
Thus $\mathbb{F}^n$ is an abelian group with respect to addition. \\
\vfill
Futher we observe that
\begin{enumerate}
	\item If a $\in \mathbb{F}$ and $\alpha = (\alpha_1,\alpha_2,\cdots,\alpha_n), \ 
		\beta = (\beta_1,\beta_2,\cdots,\beta_n) \in \mathbb{F}^n $ then
		\begin{align}
			a(\alpha+\beta)&=a(\alpha_1+\beta_1,\alpha_2+\beta_2,\cdots,\alpha_n+\beta_n) \\
			&=(a[\alpha_1+\beta_1],a[\alpha_2+\beta_2],\cdots,a[\alpha_n+\beta_n]) \\
			&=(a\alpha_1+a\beta_1,a\alpha_2+a\beta_2,\cdots,a\alpha_n+a\beta_n) \\
			&=(a\alpha_1,a\alpha_2,\cdots,a\alpha_n)+(a\beta_1,a\beta_2,\cdots,a\beta_n) \\
			&=a(\alpha_1,\alpha_2,\cdots,\alpha_n)+a(\beta_1,\beta_2,\cdots,\beta_n) \\
			&=a\alpha+a\beta
		\end{align}
	\item If a,b $\in \mathbb{F}$ and $(\alpha_1,\alpha_2,\cdots,\alpha_n) \in \mathbb{F}^n$ then

		\begin{align}
			(a+b)\alpha&=([a+b]\alpha_1,[a+b]\alpha_2,\cdots,[a+b]\alpha_n) 
		\end{align}
		\begin{align}
                        &=(a\alpha_1+b\alpha_1,a\alpha_2+b\alpha_2,\cdots,a\alpha_n+b\alpha_n) \\ 
                        &=(a\alpha_1,a\alpha_2,\cdots,a\alpha_n)+(b\alpha_1,b\alpha_2,\cdots,b\alpha_n) \\
                        &=a(\alpha_1,\alpha_2,\cdots,\alpha_n)+b(\alpha_1,\alpha_2,\cdots,\alpha_n) \\
                        &=a\alpha+b\alpha
		\end{align}
	\item If a,b $\in \mathbb{F}$ and $(\alpha_1,\alpha_2,\cdots,\alpha_n) \in \mathbb{F}^n$ then
	\begin{align}
		(ab)\alpha&=([ab]\alpha_1,[ab]\alpha_2,\cdots,[ab]\alpha_n) \\
			  &=(a[b\alpha_1],a[b\alpha_2],\cdots,a[b\alpha_n]) \\
			  &=a(b\alpha_1,b\alpha_2,\cdots,b\alpha_n) \\
			  &=a(b\alpha)
	\end{align}
        \item If 1 is the unity element of $\mathbb{F}$ and $\alpha=(\alpha_1,\alpha_2,\alpha_n) \in 
		\mathbb{F}^n$ then
		\begin{align}
			1\alpha &=(1\alpha_1,1\alpha_2,\cdots,1\alpha_n) \\
			        &=(\alpha_1,\alpha_2,\cdots,\alpha_n) \\
				&= \alpha
		\end{align}
\end{enumerate}
Hence $\mathbb{F}^n$ is a vector space over $\mathbb{F}$.
\end{document}
