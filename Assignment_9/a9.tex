\documentclass[journal,12pt,twocolumn]{IEEEtran}

\usepackage{setspace}
\usepackage{gensymb}

\singlespacing


\usepackage[cmex10]{amsmath}

\usepackage{amsthm}

\usepackage{mathrsfs}
\usepackage{txfonts}
\usepackage{stfloats}
\usepackage{bm}
\usepackage{cite}
\usepackage{cases}
\usepackage{subfig}

\usepackage{longtable}
\usepackage{multirow}

\usepackage{enumitem}
\usepackage{mathtools}
%\usepackage{steinmetz}
\usepackage{tikz}
\usepackage{circuitikz}
\usepackage{verbatim}
%\usepackage{tfrupee}
\usepackage[breaklinks=true]{hyperref}

\usepackage{tkz-euclide}

\usetikzlibrary{calc,math}
\usepackage{listings}
    \usepackage{color}                                            %%
    \usepackage{array}                                            %%
    \usepackage{longtable}                                        %%
    \usepackage{calc}                                             %%
    \usepackage{multirow}                                         %%
    \usepackage{hhline}                                           %%
    \usepackage{ifthen}                                           %%
    \usepackage{lscape}     
\usepackage{multicol}
\usepackage{chngcntr}

\DeclareMathOperator*{\Res}{Res}

\renewcommand\thesection{\arabic{section}}
\renewcommand\thesubsection{\thesection.\arabic{subsection}}
\renewcommand\thesubsubsection{\thesubsection.\arabic{subsubsection}}

\renewcommand\thesectiondis{\arabic{section}}
\renewcommand\thesubsectiondis{\thesectiondis.\arabic{subsection}}
\renewcommand\thesubsubsectiondis{\thesubsectiondis.\arabic{subsubsection}}


\hyphenation{op-tical net-works semi-conduc-tor}
\def\inputGnumericTable{}                                 %%

\lstset{
%language=C,
frame=single, 
breaklines=true,
columns=fullflexible
}
\begin{document}


\newtheorem{theorem}{Theorem}[section]
\newtheorem{problem}{Problem}
\newtheorem{proposition}{Proposition}[section]
\newtheorem{lemma}{Lemma}[section]
\newtheorem{corollary}[theorem]{Corollary}
\newtheorem{example}{Example}[section]
\newtheorem{definition}[problem]{Definition}

\newcommand{\BEQA}{\begin{eqnarray}}
\newcommand{\EEQA}{\end{eqnarray}}
\newcommand{\define}{\stackrel{\triangle}{=}}
\bibliographystyle{IEEEtran}
\providecommand{\mbf}{\mathbf}
\providecommand{\pr}[1]{\ensuremath{\Pr\left(#1\right)}}
\providecommand{\qfunc}[1]{\ensuremath{Q\left(#1\right)}}
\providecommand{\sbrak}[1]{\ensuremath{{}\left[#1\right]}}
\providecommand{\lsbrak}[1]{\ensuremath{{}\left[#1\right.}}
\providecommand{\rsbrak}[1]{\ensuremath{{}\left.#1\right]}}
\providecommand{\brak}[1]{\ensuremath{\left(#1\right)}}
\providecommand{\lbrak}[1]{\ensuremath{\left(#1\right.}}
\providecommand{\rbrak}[1]{\ensuremath{\left.#1\right)}}
\providecommand{\cbrak}[1]{\ensuremath{\left\{#1\right\}}}
\providecommand{\lcbrak}[1]{\ensuremath{\left\{#1\right.}}
\providecommand{\rcbrak}[1]{\ensuremath{\left.#1\right\}}}
\theoremstyle{remark}
\newtheorem{rem}{Remark}
\newcommand{\sgn}{\mathop{\mathrm{sgn}}}
\providecommand{\abs}[1]{\left\vert#1\right\vert}
\providecommand{\res}[1]{\Res\displaylimits_{#1}} 
\providecommand{\norm}[1]{\left\lVert#1\right\rVert}
%\providecommand{\norm}[1]{\lVert#1\rVert}
\providecommand{\mtx}[1]{\mathbf{#1}}
\providecommand{\mean}[1]{E\left[ #1 \right]}
\providecommand{\fourier}{\overset{\mathcal{F}}{ \rightleftharpoons}}
%\providecommand{\hilbert}{\overset{\mathcal{H}}{ \rightleftharpoons}}
\providecommand{\system}{\overset{\mathcal{H}}{ \longleftrightarrow}}
	%\newcommand{\solution}[2]{\textbf{Solution:}{#1}}
\newcommand{\solution}{\noindent \textbf{Solution: }}
\newcommand{\cosec}{\,\text{cosec}\,}
\providecommand{\dec}[2]{\ensuremath{\overset{#1}{\underset{#2}{\gtrless}}}}
\newcommand{\myvec}[1]{\ensuremath{\begin{pmatrix}#1\end{pmatrix}}}
\newcommand{\mydet}[1]{\ensuremath{\begin{vmatrix}#1\end{vmatrix}}}
\numberwithin{equation}{subsection}
\makeatletter
\@addtoreset{figure}{problem}
\makeatother
\let\StandardTheFigure\thefigure
\let\vec\mathbf
\renewcommand{\thefigure}{\theproblem}
\def\putbox#1#2#3{\makebox[0in][l]{\makebox[#1][l]{}\raisebox{\baselineskip}[0in][0in]{\raisebox{#2}[0in][0in]{#3}}}}
     \def\rightbox#1{\makebox[0in][r]{#1}}
     \def\centbox#1{\makebox[0in]{#1}}
     \def\topbox#1{\raisebox{-\baselineskip}[0in][0in]{#1}}
     \def\midbox#1{\raisebox{-0.5\baselineskip}[0in][0in]{#1}}
\vspace{3cm}
\title{EE5609 Assignment 9}
\author{SHANTANU YADAV, EE20MTECH12001 }
\maketitle
\newpage
\bigskip
\renewcommand{\thefigure}{\theenumi}
\renewcommand{\thetable}{\theenumi}

The python solution code is available at
\begin{lstlisting}
https://github.com/Shantanu2508/Matrix_Theory/blob/master/Assignment_9/assignment9.py
\end{lstlisting}

\section{Problem}
Discover whether
\begin{align}
	\vec{A}=\myvec{ 1 & 2 & 3 & 4 \\
			0 & 2 & 3 & 4 \\
			0 & 0 & 3 & 4 \\
			0 & 0 & 0 & 4}
\end{align}
is invertible, and find $\vec{A^{-1}}$ if it exists.
\section{Solution}
The matrix $\vec{A}$ is in row reduced echolon form with four pivot elements. Therefore the
rank($\vec{A}$) is 4. Hence the rows of matrix $\vec{A}$ constitute of 4 linearly independent
vectors. Thus it can be concluded that matrix $\vec{A}$ is invertible. Using Gauss-Jordan Elimination,
if there exists an elimentary matrix $\vec{E}$ such that $\vec{E[A \ I]=[I \ E]}$ then $\vec{E}$ is the inverse of A i.e $\vec{E}=\vec{A^{-1}}$.
\begin{align}
	\vec{[A \ I]} = \myvec{ 1 & 2 & 3 & 4 &| & 1 & 0 & 0 & 0 \\
				0 & 2 & 3 & 4 &| & 0 & 1 & 0 & 0 \\
				0 & 0 & 3 & 4 &| & 0 & 0 & 1 & 0 \\
				0 & 0 & 0 & 4 &| & 0 & 0 & 0 & 1 }\\
			\xleftrightarrow[]{R_1 \leftarrow R_1 - R_2}
			\myvec{ 1 & 0 & 0 & 0 &| & 1 & -1 & 0 & 0 \\
                                0 & 2 & 3 & 4 &| & 0 & 1 & 0 & 0 \\
                                0 & 0 & 3 & 4 &| & 0 & 0 & 1 & 0 \\
                                0 & 0 & 0 & 4 &| & 0 & 0 & 0 & 1 }\\
			\xleftrightarrow[]{R_2 \leftarrow R_2 - R_3}
                        \myvec{ 1 & 0 & 0 & 0 &| & 1 & -1 & 0 & 0 \\
                                0 & 2 & 0 & 0 &| & 0 & 1 & -1 & 0 \\
                                0 & 0 & 3 & 4 &| & 0 & 0 & 1 & 0 \\
                                0 & 0 & 0 & 4 &| & 0 & 0 & 0 & 1 }\\
			\xleftrightarrow[]{R_3 \leftarrow R_3 - R_4}
                        \myvec{ 1 & 0 & 0 & 0 &| & 1 & -1 & 0 & 0 \\
                                0 & 2 & 0 & 0 &| & 0 & 1 & -1 & 0 \\
                                0 & 0 & 3 & 0 &| & 0 & 0 & 1 & -1 \\
                                0 & 0 & 0 & 4 &| & 0 & 0 & 0 & 1 }\\
			\xleftrightarrow[R_2 \leftarrow \frac{R_2}{2} \ R_3 \leftarrow \frac{R_3}{3}]
			{R_4 \leftarrow \frac{R_4}{4}}
                        \myvec{ 1 & 0 & 0 & 0 &| & 1 & -1 & 0 & 0 \\
				0 & 1 & 0 & 0 &| & 0 & \frac{1}{2} & -\frac{1}{2} & 0 \\
				0 & 0 & 1 & 0 &| & 0 & 0 & \frac{1}{3} & -\frac{1}{3} \\
				0 & 0 & 0 & 1 &| & 0 & 0 & 0 & \frac{1}{4}} \\ 
			= \vec{[I \ E]}
\end{align}
Therefore $\vec{A^{-1}}$ =
\begin{align}
	\myvec{ 1 & -1 & 0 & 0 \\
                0 & \frac{1}{2} & -\frac{1}{2} & 0 \\
                0 & 0 & \frac{1}{3} & -\frac{1}{3} \\
                0 & 0 & 0 & \frac{1}{4}} 
\end{align}
\end{document}
