\documentclass[journal,12pt,twocolumn]{IEEEtran}

\usepackage{setspace}
\usepackage{gensymb}

\singlespacing


\usepackage[cmex10]{amsmath}

\usepackage{amsthm}

\usepackage{mathrsfs}
\usepackage{txfonts}
\usepackage{stfloats}
\usepackage{bm}
\usepackage{cite}
\usepackage{cases}
\usepackage{subfig}

\usepackage{longtable}
\usepackage{multirow}

\usepackage{enumitem}
\usepackage{mathtools}
%\usepackage{steinmetz}
\usepackage{tikz}
\usepackage{circuitikz}
\usepackage{verbatim}
%\usepackage{tfrupee}
\usepackage[breaklinks=true]{hyperref}

\usepackage{tkz-euclide}

\usetikzlibrary{calc,math}
\usepackage{listings}
    \usepackage{color}                                            %%
    \usepackage{array}                                            %%
    \usepackage{longtable}                                        %%
    \usepackage{calc}                                             %%
    \usepackage{multirow}                                         %%
    \usepackage{hhline}                                           %%
    \usepackage{ifthen}                                           %%
    \usepackage{lscape}     
\usepackage{multicol}
\usepackage{chngcntr}
%\usepackage{enumerate,float}
\DeclareMathOperator*{\Res}{Res}

\renewcommand\thesection{\arabic{section}}
\renewcommand\thesubsection{\thesection.\arabic{subsection}}
\renewcommand\thesubsubsection{\thesubsection.\arabic{subsubsection}}

\renewcommand\thesectiondis{\arabic{section}}
\renewcommand\thesubsectiondis{\thesectiondis.\arabic{subsection}}
\renewcommand\thesubsubsectiondis{\thesubsectiondis.\arabic{subsubsection}}


\hyphenation{op-tical net-works semi-conduc-tor}
\def\inputGnumericTable{}                                 %%

\lstset{
%language=C,
frame=single, 
breaklines=true,
columns=fullflexible
}
\begin{document}


\newtheorem{theorem}{Theorem}[section]
\newtheorem{problem}{Problem}
\newtheorem{proposition}{Proposition}[section]
\newtheorem{lemma}{Lemma}[section]
\newtheorem{corollary}[theorem]{Corollary}
\newtheorem{example}{Example}[section]
\newtheorem{definition}[problem]{Definition}

\newcommand{\BEQA}{\begin{eqnarray}}
\newcommand{\EEQA}{\end{eqnarray}}
\newcommand{\define}{\stackrel{\triangle}{=}}
\bibliographystyle{IEEEtran}
\providecommand{\mbf}{\mathbf}
\providecommand{\pr}[1]{\ensuremath{\Pr\left(#1\right)}}
\providecommand{\qfunc}[1]{\ensuremath{Q\left(#1\right)}}
\providecommand{\sbrak}[1]{\ensuremath{{}\left[#1\right]}}
\providecommand{\lsbrak}[1]{\ensuremath{{}\left[#1\right.}}
\providecommand{\rsbrak}[1]{\ensuremath{{}\left.#1\right]}}
\providecommand{\brak}[1]{\ensuremath{\left(#1\right)}}
\providecommand{\lbrak}[1]{\ensuremath{\left(#1\right.}}
\providecommand{\rbrak}[1]{\ensuremath{\left.#1\right)}}
\providecommand{\cbrak}[1]{\ensuremath{\left\{#1\right\}}}
\providecommand{\lcbrak}[1]{\ensuremath{\left\{#1\right.}}
\providecommand{\rcbrak}[1]{\ensuremath{\left.#1\right\}}}
\theoremstyle{remark}
\newtheorem{rem}{Remark}
\newcommand{\sgn}{\mathop{\mathrm{sgn}}}
\providecommand{\abs}[1]{\left\vert#1\right\vert}
\providecommand{\res}[1]{\Res\displaylimits_{#1}} 
\providecommand{\norm}[1]{\left\lVert#1\right\rVert}
%\providecommand{\norm}[1]{\lVert#1\rVert}
\providecommand{\mtx}[1]{\mathbf{#1}}
\providecommand{\mean}[1]{E\left[ #1 \right]}
\providecommand{\fourier}{\overset{\mathcal{F}}{ \rightleftharpoons}}
%\providecommand{\hilbert}{\overset{\mathcal{H}}{ \rightleftharpoons}}
\providecommand{\system}{\overset{\mathcal{H}}{ \longleftrightarrow}}
	%\newcommand{\solution}[2]{\textbf{Solution:}{#1}}
\newcommand{\solution}{\noindent \textbf{Solution: }}
\newcommand{\cosec}{\,\text{cosec}\,}
\providecommand{\dec}[2]{\ensuremath{\overset{#1}{\underset{#2}{\gtrless}}}}
\newcommand{\myvec}[1]{\ensuremath{\begin{pmatrix}#1\end{pmatrix}}}
\newcommand{\mydet}[1]{\ensuremath{\begin{vmatrix}#1\end{vmatrix}}}
\numberwithin{equation}{subsection}
\makeatletter
\@addtoreset{figure}{problem}
\makeatother
\let\StandardTheFigure\thefigure
\let\vec\mathbf
\renewcommand{\thefigure}{\theproblem}
\def\putbox#1#2#3{\makebox[0in][l]{\makebox[#1][l]{}\raisebox{\baselineskip}[0in][0in]{\raisebox{#2}[0in][0in]{#3}}}}
     \def\rightbox#1{\makebox[0in][r]{#1}}
     \def\centbox#1{\makebox[0in]{#1}}
     \def\topbox#1{\raisebox{-\baselineskip}[0in][0in]{#1}}
     \def\midbox#1{\raisebox{-0.5\baselineskip}[0in][0in]{#1}}
\vspace{3cm}
\title{EE5609 Assignment 13}
\author{SHANTANU YADAV, EE20MTECH12001 }
\maketitle
\newpage
\bigskip
\renewcommand{\thefigure}{\theenumi}
\renewcommand{\thetable}{\theenumi}

\section{Problem}
\begin{enumerate}[label=\emph{\alph*)}]
\item
Let $\vec{F}$ be a field and let $\vec{V}$ be the space of polynomial functions $f$ from $\vec{F}$
into $\vec{F}$, given by
\begin{align*}
	f(x)=c_0+c_1x+\cdots +c_nx^n
\end{align*}
Let $\vec{D}$ be a linear differentiation transformation defined as
\begin{align*}
	(\vec{D}f)(x)=\frac{df(x)}{dx}
\end{align*}
Then find the range and null space of 
$\vec{D}$.\\
\item
Let $\vec{R}$ be the field of real numbers and let $\vec{V}$ be the space of all functions from 
$\vec{R}$ into $\vec{R}$ which are continous. Let $\vec{T}$ be linear transformation defined by
\begin{align*}
	(\vec{T}f)(x)=\int_{0}^{x} f(t)\,dt
\end{align*}
Find the range and null space of $\vec{T}$.
\end{enumerate}
\section{Explanation}
Let the vector space of n-dimension be deined as
\begin{align}
        \vec{V}=\cbrak{f:\vec{F}\rightarrow \vec{F} : f(x)=\sum_{k=0}^{n}c_kx^k, \ c_k \in \vec{F} }
\end{align}
The corresponding standard basis for $\vec{V}$ is
\begin{align}
	\cbrak{\myvec{1\\0\\\vdots\\0},\myvec{0\\x\\\vdots\\0},\cdots,\myvec{0\\0\\\vdots\\x^{n-1}}}
\end{align}
\begin{enumerate}[label=\emph{\alph*)}]
\item
Let $f$ and $g \in \vec{V}$ and let $\alpha$ and $\beta \in \vec{F}$ then 
		\begin{align}
			\vec{D}(\alpha f + \beta g)&=\frac{d(\alpha f(x) + \beta g(x))}{dx} \\
			&=\alpha\frac{df(x)}{dx}+\beta\frac{dg(x)}{dx}\\
			&=\alpha(\vec{D}f)+\beta(\vec{D}g)
		\end{align}
Therefore $\vec{D}$ is a linear transformation.\\
The $\vec{D}$ transformation maps the $k^{th}$ basis vector as follows
		\begin{align}
			\vec{D}\myvec{0\\\vdots\\0\\x^k\\\vdots\\0}
			=\myvec{0\\\vdots\\kx^{k-1}\\0\\\vdots\\0}
		\end{align}
Since the transformed vector 
		\begin{align}
			\myvec{0\\\vdots\\kx^{k-1}\\0\\\vdots\\0} \in \vec{V}
		\end{align}
the range of $\vec{D}$ is the vector space $\vec{V}$. Thus the transformation is defined as
$\vec{D}:\vec{V} \rightarrow \vec{V}$.
Therefore the $\vec{D}$ Transformation on the basis vector set is
		\begin{align}
			\vec{D}\myvec{1 & 0 & 0 & \cdots & 0 & 0\\
				      0 & 1 & 0 & \cdots & 0 & 0 \\
				      0 & 0 & 1 & \cdots & 0 & 0 \\
				      \vdots & \vdots & \vdots & \vdots & \vdots & \vdots\\
				      0 & 0 & 0 & \cdots & 1 & 0 \\
				      0 & 0 & 0 & \cdots & 0 & 1}\\
			      =\myvec{0 & 1 & 0 & \cdots & 0 & 0\\
				      0 & 0 & 2 & \cdots & 0 & 0\\
				      0 & 0 & 0 & \cdots & 0 & 0\\
				      \vdots & \vdots & \vdots & \vdots & \vdots & \vdots\\
				      0 & 0 & 0 & \cdots & 0 & n-2 \\
				      0 & 0 & 0 & \cdots & 0 & 0}
		\end{align}
Thus the $\vec{D}$ transformation coefficient matrix is		
\begin{align}
	D=\myvec{0 & 1 & 0 & \cdots & 0 & 0\\
                 0 & 0 & 2 & \cdots & 0 & 0\\
                 0 & 0 & 0 & \cdots & 0 & 0\\
                \vdots & \vdots & \vdots & \vdots & \vdots & \vdots\\
                 0 & 0 & 0 & \cdots & 0 & n-2 \\
                 0 & 0 & 0 & \cdots & 0 & 0}
\end{align}
Since $D$ contains a zero row hence $\mydet{D}=0$. Therefore $\vec{D}$ transformation matrix is 
singular. The nullspace for differentiation transformation is defined as
\begin{align}
        \vec{N}=\cbrak{f \in \vec{V} : \vec{D}f=0} 
\end{align}
		Let the coefficient matrix of $f \in \vec{V}$ be 
		\begin{align}
			\vec{f}=\myvec{c_0\\c_1\\\vdots\\c_{n-1}}
		\end{align}
		then
		\begin{align}
			\vec{D}f&=0 \\
			\implies
			&\myvec{0 & 1 & 0 & \cdots & 0 & 0\\
                 0 & 0 & 2 & \cdots & 0 & 0\\
                 0 & 0 & 0 & \cdots & 0 & 0\\
                \vdots & \vdots & \vdots & \vdots & \vdots & \vdots\\
                 0 & 0 & 0 & \cdots & 0 & n-2 \\
		0 & 0 & 0 & \cdots & 0 & 0}\myvec{c_0\\c_1\\\vdots\\c_{n-1}}=\vec{0} \label{dx}
		\end{align}
Since $D$ is in row reduced echolon form and $rank(D)=n-1$ the solution of (\ref{dx}) is
\begin{align}
\vec{f}=\myvec{k\\0\\\vdots\\0}
\end{align}
where $k \in \vec{R}$. Therefore the nullspace for \\$\vec{D}:\vec{V}\rightarrow\vec{V}$ is
\begin{align}
	\vec{N}=\cbrak{\myvec{k\\0\\\vdots\\0}:k \in \vec{R}}
\end{align}
\item
Let $f$ and $g \in \vec{V}$ and let $\alpha$ and $\beta \in \vec{F}$ then
\begin{align}
	\vec{T}(\alpha f + \beta g)&=\int_{0}^{x}(\alpha f(t) + \beta g(t))\,dt\\
	&=\alpha\int_{0}^{x} f(t)\,dt+\beta\int_{0}^{x} g(t)\,dt\\
     &=\alpha(\vec{T}f)+\beta(\vec{T}g)
\end{align}
Therefore $\vec{T}$ is a linear transformation.\\
The $\vec{T}$ transformation maps the $k^{th}$ basis vector as follows
\begin{align}
      \vec{T}\myvec{0\\ \vdots \\ x^k \\0 \\ \vdots \\0}
      =\myvec{0\\ \vdots \\ 0\\ \frac{x^{k+1}}{k+1} \\ \vdots \\0}
\end{align}
Since the transformed vector
\begin{align}
	\myvec{0\\ \vdots \\ 0\\ \frac{x^{k+1}}{k+1} \\ \vdots \\0} \in \vec{V}
\end{align}
the range of $\vec{T}$ is the vector space $\vec{V}$. Thus the transformation is defined as 
$\vec{T} : \vec{V}\rightarrow \vec{V}$.
Therefore the $\vec{T}$ Transformation on the basis vector set is
                \begin{align}
                        \vec{T}\myvec{1 & 0 & 0 & \cdots & 0 & 0\\
                                      0 & 1 & 0 & \cdots & 0 & 0 \\
                                      0 & 0 & 1 & \cdots & 0 & 0 \\
                                      \vdots & \vdots & \vdots & \vdots & \vdots & \vdots\\
                                      0 & 0 & 0 & \cdots & 1 & 0 \\
                                      0 & 0 & 0 & \cdots & 0 & 1} \\
                              =\myvec{0 & 0 & 0 & \cdots & 0 & 0\\
                                      1 & 0 & 0 & \cdots & 0 & 0\\
				      0 & \frac{1}{2} & 0 & \cdots & 0 & 0\\
                                      \vdots & \vdots & \vdots & \vdots & \vdots & \vdots\\
				      0 & 0 & 0 & \cdots & 0 & 0 \\
				      0 & 0 & 0 & \cdots & \frac{1}{n-1} & 0 \\
				      0 & 0 & 0 & \cdots & 0 & \frac{1}{n} }
                \end{align}
Thus the $\vec{T}$ transformation coefficient matrix is
\begin{align}
	                     T=\myvec{0 & 0 & 0 & \cdots & 0 & 0\\
                                      1 & 0 & 0 & \cdots & 0 & 0\\
                                      0 & \frac{1}{2} & 0 & \cdots & 0 & 0\\
                                      \vdots & \vdots & \vdots & \vdots & \vdots & \vdots\\
                                      0 & 0 & 0 & \cdots & 0 & 0 \\
                                      0 & 0 & 0 & \cdots & \frac{1}{n-1} & 0 \\
                                      0 & 0 & 0 & \cdots & 0 & \frac{1}{n} } 
\end{align}
Since $T$ contains a zero row hence $\mydet{T}=0$. Therefore $\vec{T}$ transformation matrix is 
singular.
The nullspace for integration transformation is defined as
\begin{align}
        \vec{N}=\cbrak{f \in \vec{V} : \vec{T}f=0}
\end{align}
                Let the coefficient matrix of $f \in \vec{V}$ be
                \begin{align}
                        \vec{f}=\myvec{c_0\\c_1\\\vdots\\c_{n-1}}
                \end{align}
                then
\begin{align}
	&\vec{T}f=0 \\
	\implies 
	&\myvec{0 & 0 & 0 & \cdots & 0 & 0\\
               1 & 0 & 0 & \cdots & 0 & 0\\
               0 & \frac{1}{2} & 0 & \cdots & 0 & 0\\
               \vdots & \vdots & \vdots & \vdots & \vdots & \vdots\\
               0 & 0 & 0 & \cdots & 0 & 0 \\
               0 & 0 & 0 & \cdots & \frac{1}{n-1} & 0 \\
               0 & 0 & 0 & \cdots & 0 & \frac{1}{n} }
	       \myvec{c_0\\c_1\\\vdots\\c_{n-1}}=\vec{0} \label{tx}
\end{align}
Since $T$ is in row reduced echolon form and $rank(T)=n$ the solution of (\ref{tx}) is
\begin{align}
\vec{f}=\myvec{0\\0\\\vdots\\0}
\end{align}
where $k \in \vec{R}$. Therefore the nullspace for \\$\vec{T}:\vec{V}\rightarrow\vec{V}$ is
\begin{align}
        \vec{N}=\cbrak{\myvec{0\\0\\\vdots\\0}:k \in \vec{R}}
\end{align}
\end{enumerate}
\end{document}
