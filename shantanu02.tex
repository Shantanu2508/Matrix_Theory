%
%--------------------------------------------------------
%  Assignment 1
%  Shantanu Yadav, EE20MTech12001
%  IIT Hyderabad
%--------------------------------------------------------
%
%\documentclass[12pt]{IEEEtran}
\documentclass[12pt]{article}
\usepackage{amsmath}
\usepackage{graphicx}
\usepackage{tabularx}
\usepackage{listings}
\lstset{frame=single,breaklines=true,columns=fullflexible}     % this is used for making box boundary of lslisting

\textheight=9.5in
\textwidth=6.0in
\topmargin=-0.5in
\newcommand{\myvec}[1]{\ensuremath{\begin{pmatrix}#1\end{pmatrix}}} % this is used for creating vector and matrix
\thispagestyle{empty} % this is used for avoiding page no. on current page

\begin{document}
\begin{center}
	{\large \bf EE5609 ASSIGNMENT 1} \\
	\vspace{2ex}
	{\large \bf SHANTANU YADAV, EE20MTECH12001 }\\
	{\Large \bf IIT Hyderabad} \\ \vspace{2ex}
\end{center}
	\hrule

\vspace{2ex}
\begin{center}
{\underline{\Large \bf Lines and Planes}}
\end{center}

%---------------------------------------------------------
\noindent 
The python solution code is available at
\begin{lstlisting}
	https://github.com/Shantanu2508/Matrix_Theory/blob/master/stline.py
\end{lstlisting}
%---------------------------------------------------------

\section*{Problem Statement}
Find the equations of the lines which intercepts on the both the axes and whose sum and product are 1 and -6 respectively.

\section*{Solution}
The equation of line in terms of vector notations can be written as
\begin{align}
	{\mathbf{n}^T}{\mathbf x} = c  
\end{align}
Let the intercepts be $\myvec{a \\ 0}$ and $\myvec{0 \\ b}$, respectively.

\noindent
Given that: \qquad \quad $ a + b = 1 $, \qquad and \qquad \quad $ ab = -6$ \\
The quadratic equation whose roots are the x and y intercepts can be written 
as :
\begin{align*}
	x^2 - (\text{sum of roots})x + (\text{product of roots}) = 0
\end{align*}
\begin{align}
	\implies \qquad x^2 - x -6 =0
\end{align}
\begin{align*}
\implies \quad x=(3,-2) \quad \text{and corresponding $y$ intercepts are  } (-2,3)
\end{align*}

\noindent
The line L1 passes through $\myvec{3 \\ 0}$ and $\myvec{0 \\ -2}$. \\

%............................................................................
\noindent Let direction vector of this line be $\mathbf{m}$.
\begin{align*}
	\mathbf{m} = \myvec{0\\-2} - \myvec{3\\0} = \myvec{-3\\-2}
\end{align*}
The normal vector, $\mathbf{n}$: 
\begin{align*}
	\mathbf{n} = \myvec{0 & -1 \\ 1 & 0}\mathbf{m} = \myvec{2 \\ -3}
\end{align*}
%............................................................................
The equation of line in terms of normal vector and passing through a point 
$A$ is
\begin{align*}
	\quad \mathbf{n^T}(\mathbf{x} - A ) = 0
	\qquad \implies \qquad \mathbf{n^Tx} = \mathbf{n^T}A
\end{align*}
\begin{align*}
	\implies \mathbf{n^Tx}= \myvec{2 -3}\myvec{3 \\ 0}
\end{align*}
\begin{align}
	\implies \myvec{2 -3}\mathbf{x} = 6	\label{eqL1}
\end{align}
%............................................................................
\begin{figure}[htbp]
\centering
\includegraphics[width=0.7\linewidth]{stline_plot.png}
\caption{}
\end{figure}

%............................................................................
\noindent Similarly, the equation of second line L2, with the $x$ and 
$y$ intercepts $\myvec{ -2 \\ 0}$ and $\myvec{ 0 \\ 3}$ and 
normal vector $\myvec{ -3 \\ 2}$ is
\begin{align}
	\myvec{-3 &  2 } {\mathbf{x}} = 6	\label{eqL2}
\end{align}
The equations of lines (\ref{eqL1}) and (\ref{eqL2}) can be represented 
collectively as
\begin{align}
	\myvec{2 & -3 \\ -3 & 2}\mathbf{x} = \myvec{6 \\ 6}
\end{align}
%..............................................................................
\begin{table}[htbp]
	\centering
	\begin{tabularx}{0.5\linewidth}{|c|c|X|} \hline 
	$x-$intercept & $y-$intercept & \quad $\mathbf{n}$  \\ \hline \hline
	\myvec{3\\0}	&	\myvec{0\\-2}	&	\myvec{2\\-3} \\ \hline
	\myvec{-2\\0}	&	\myvec{0\\3}	&	\myvec{-3\\2} \\ \hline
	\end{tabularx}
	\caption{}
\end{table}
%..............................................................................
\end{document}
