\documentclass[journal,12pt,twocolumn]{IEEEtran}

\usepackage{setspace}
\usepackage{gensymb}

\singlespacing


\usepackage[cmex10]{amsmath}

\usepackage{amsthm}

\usepackage{mathrsfs}
\usepackage{txfonts}
\usepackage{stfloats}
\usepackage{bm}
\usepackage{cite}
\usepackage{cases}
\usepackage{subfig}

\usepackage{longtable}
\usepackage{multirow}

\usepackage{enumitem}
\usepackage{mathtools}
%\usepackage{steinmetz}
\usepackage{tikz}
\usepackage{circuitikz}
\usepackage{verbatim}
%\usepackage{tfrupee}
\usepackage[breaklinks=true]{hyperref}

\usepackage{tkz-euclide}

\usetikzlibrary{calc,math}
\usepackage{listings}
    \usepackage{color}                                            %%
    \usepackage{array}                                            %%
    \usepackage{longtable}                                        %%
    \usepackage{calc}                                             %%
    \usepackage{multirow}                                         %%
    \usepackage{hhline}                                           %%
    \usepackage{ifthen}                                           %%
    \usepackage{lscape}     
\usepackage{multicol}
\usepackage{chngcntr}
%\usepackage{enumerate,float}
\DeclareMathOperator*{\Res}{Res}

\renewcommand\thesection{\arabic{section}}
\renewcommand\thesubsection{\thesection.\arabic{subsection}}
\renewcommand\thesubsubsection{\thesubsection.\arabic{subsubsection}}

\renewcommand\thesectiondis{\arabic{section}}
\renewcommand\thesubsectiondis{\thesectiondis.\arabic{subsection}}
\renewcommand\thesubsubsectiondis{\thesubsectiondis.\arabic{subsubsection}}


\hyphenation{op-tical net-works semi-conduc-tor}
\def\inputGnumericTable{}                                 %%

\lstset{
%language=C,
frame=single, 
breaklines=true,
columns=fullflexible
}
\begin{document}


\newtheorem{theorem}{Theorem}[section]
\newtheorem{problem}{Problem}
\newtheorem{proposition}{Proposition}[section]
\newtheorem{lemma}{Lemma}[section]
\newtheorem{corollary}[theorem]{Corollary}
\newtheorem{example}{Example}[section]
\newtheorem{definition}[problem]{Definition}

\newcommand{\BEQA}{\begin{eqnarray}}
\newcommand{\EEQA}{\end{eqnarray}}
\newcommand{\define}{\stackrel{\triangle}{=}}
\bibliographystyle{IEEEtran}
\providecommand{\mbf}{\mathbf}
\providecommand{\pr}[1]{\ensuremath{\Pr\left(#1\right)}}
\providecommand{\qfunc}[1]{\ensuremath{Q\left(#1\right)}}
\providecommand{\sbrak}[1]{\ensuremath{{}\left[#1\right]}}
\providecommand{\lsbrak}[1]{\ensuremath{{}\left[#1\right.}}
\providecommand{\rsbrak}[1]{\ensuremath{{}\left.#1\right]}}
\providecommand{\brak}[1]{\ensuremath{\left(#1\right)}}
\providecommand{\lbrak}[1]{\ensuremath{\left(#1\right.}}
\providecommand{\rbrak}[1]{\ensuremath{\left.#1\right)}}
\providecommand{\cbrak}[1]{\ensuremath{\left\{#1\right\}}}
\providecommand{\lcbrak}[1]{\ensuremath{\left\{#1\right.}}
\providecommand{\rcbrak}[1]{\ensuremath{\left.#1\right\}}}
\theoremstyle{remark}
\newtheorem{rem}{Remark}
\newcommand{\sgn}{\mathop{\mathrm{sgn}}}
\providecommand{\abs}[1]{\left\vert#1\right\vert}
\providecommand{\res}[1]{\Res\displaylimits_{#1}} 
\providecommand{\norm}[1]{\left\lVert#1\right\rVert}
%\providecommand{\norm}[1]{\lVert#1\rVert}
\providecommand{\mtx}[1]{\mathbf{#1}}
\providecommand{\mean}[1]{E\left[ #1 \right]}
\providecommand{\fourier}{\overset{\mathcal{F}}{ \rightleftharpoons}}
%\providecommand{\hilbert}{\overset{\mathcal{H}}{ \rightleftharpoons}}
\providecommand{\system}{\overset{\mathcal{H}}{ \longleftrightarrow}}
	%\newcommand{\solution}[2]{\textbf{Solution:}{#1}}
\newcommand{\solution}{\noindent \textbf{Solution: }}
\newcommand{\cosec}{\,\text{cosec}\,}
\providecommand{\dec}[2]{\ensuremath{\overset{#1}{\underset{#2}{\gtrless}}}}
\newcommand{\myvec}[1]{\ensuremath{\begin{pmatrix}#1\end{pmatrix}}}
\newcommand{\mydet}[1]{\ensuremath{\begin{vmatrix}#1\end{vmatrix}}}
\numberwithin{equation}{subsection}
\makeatletter
\@addtoreset{figure}{problem}
\makeatother
\let\StandardTheFigure\thefigure
\let\vec\mathbf
\renewcommand{\thefigure}{\theproblem}
\def\putbox#1#2#3{\makebox[0in][l]{\makebox[#1][l]{}\raisebox{\baselineskip}[0in][0in]{\raisebox{#2}[0in][0in]{#3}}}}
     \def\rightbox#1{\makebox[0in][r]{#1}}
     \def\centbox#1{\makebox[0in]{#1}}
     \def\topbox#1{\raisebox{-\baselineskip}[0in][0in]{#1}}
     \def\midbox#1{\raisebox{-0.5\baselineskip}[0in][0in]{#1}}
\vspace{3cm}
\title{EE5609 Assignment 17}
\author{SHANTANU YADAV, EE20MTECH12001 }
\maketitle
\newpage
\bigskip
\renewcommand{\thefigure}{\theenumi}
%\renewcommand{\thetable}{\theenumi}

\section{Problem}
Let $\vec{A}$ be a $n\times n$ real matrix with $\vec{A}^2=\vec{A}$. Then
\begin{enumerate}
	\item the eigenvalues of $\vec{A}$ are either 0 or 1
	\item $\vec{A}$ is a diagonal matrix with diagonal entries 0 or 1
	\item $rank(\vec{A})=trace(\vec{A})$
	\item if $rank(\vec{I-A})=trace(\vec{I-A})$
\end{enumerate}
\section{Explanation}
\begin{table}[htbp]
        \centering
	\begin{tabular}{|m{2.0in}|m{5.0in}|} \hline
		\textbf{Objective} & \textbf{Explanation} \\ \hline
		Eigenvalues of $\vec{A}$ & Since 
		\begin{align}
			\vec{A}^2=\vec{A} \\
		        \implies \vec{A}^2-\vec{A}=\vec{O}
		\end{align} 
From Cayley-Hamilton Theorem we have,
\begin{align}
        \lambda^2-\lambda=0 \\
        \implies \lambda\brak{\lambda-1}=0 \\
        \implies \lambda=0,1
\end{align} \\ \hline
		Relation between rank and trace of $\vec{A}$ & A matrix $\vec{A}$ satisfying $\vec{A}^2=\vec{A}$ is an idempotent matrix with eigen values      
equal to 0 or 1. For such a matrix 
\begin{align}
	rank(\vec{A})=trace(\vec{A})
\end{align} \\ \hline
Relation between rank and trace of $\vec{I}-\vec{A}$ & Now for the matrix $\vec{I}-\vec{A}$ we have,
\begin{align}
	\brak{\vec{I}-\vec{A}}^2 = \brak{\vec{I}-\vec{A}}\brak{\vec{I}-\vec{A}}\\
	= \vec{I}^2-\vec{IA}-\vec{AI}+\vec{A}^2 \\
	= \vec{I}-\vec{A}-\vec{A}+\vec{A} \\
	= \vec{I}-\vec{A}
\end{align}
Hence $\vec{I}-\vec{A}$ is an idempotent matrix. Therefore we conclude,
\begin{align}
        rank(\vec{I}-\vec{A})=trace(\vec{I}-\vec{A})
\end{align} \\ \hline
        \end{tabular}
        \caption{} \label{1}
\end{table}
Thus, options (1),(3) and (4) are correct.
\end{document}
