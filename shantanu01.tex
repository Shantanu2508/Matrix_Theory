


%
%--------------------------------------------------------
%  Assignment 1
%  Shantanu Yadav, EE20MTech12001
%  IIT Hyderabad
%--------------------------------------------------------
%
\documentclass[12pt]{article}
\usepackage{amsmath}
\usepackage{graphicx}
\textheight=9.0in
\textwidth=6.0in
\thispagestyle{empty}

\begin{document}
\begin{center}
	{\Large \bf IIT Hyderabad} \\ \vspace{2ex}
	{\large \bf SHANTANU YADAV, EE20MTECH12001 }\\
	\vspace{2ex}
	{\large \bf ASSIGNMENT 1} \\
\end{center}
	\hrule

\vspace{2ex}
\begin{center}
{\underline{\Large \bf Lines and Planes}}
\end{center}

\section*{Problem Statement}
Find the equations of the lines which intercepts on the both the axes and whose sum and product are 1 and -6 respectively.

\section*{Solution}
The equation of line in terms of vector notations can be written as
\begin{equation}
	{\mathbf{n}^T}{\mathbf x} = b 
	\qquad \text{ where } \qquad 
	\mathbf{n} = 
\begin{pmatrix}
	n_{11} \\ 
	n_{12}
\end{pmatrix}
\end{equation}
	or
\begin{equation}
\begin{pmatrix}
	n_{11} & n_{12}
\end{pmatrix}
	{\mathbf{x}} = b	\label{eq2}
\end{equation}
Let the intercepts be 
$\displaystyle
\begin{pmatrix}
	a \\ 0
\end{pmatrix}$
and 
$\displaystyle
\begin{pmatrix}
	0 \\ b
\end{pmatrix}$, respectively. \\


\begin{figure}


\centering
\includegraphics[width=1.0\linewidth]{stline_plot.png}


\caption{}
\end{figure}


\noindent
Given that: \qquad \quad $ a + b = 1 $, \qquad and \qquad \quad $ ab = -6$ 

\begin{equation*}
\implies b = \frac{-6}{a} \  
\implies a^2 - a -6 =0
\implies (a-3)(a+2)=0
\end{equation*}
\begin{equation}
	\implies (a,b)=(3,-2) \text{ and } (-2,3)
\end{equation}

\noindent
When the line passes through 
$\displaystyle
\begin{pmatrix}
	3 \\ 0
\end{pmatrix}$
and 
$\displaystyle
\begin{pmatrix}
	0 \\ -2
\end{pmatrix}$, respectively,
we get, upon substitution in (\ref{eq2}):
	\[ 3 n_{11} = b \qquad \implies \qquad n_{11} = \frac{b}{3} \]
	\[-2 n_{12} = b \qquad \implies \qquad n_{12} =-\frac{b}{2} \]
Therefore, the equation of first line is
\begin{equation*}
\begin{pmatrix}
	\displaystyle \frac{b}{3} &
	\displaystyle \frac{-b}{2}
\end{pmatrix}
	{\mathbf{x}} = b
\end{equation*}
%
$\implies$
\begin{equation}
\begin{pmatrix}
	\displaystyle \frac{1}{3} &
	\displaystyle \frac{-1}{2}
\end{pmatrix}
	{\mathbf{x}} = 1
\end{equation}
Similarly, the equation of second line, which passes through 
$\displaystyle
\begin{pmatrix}
	-2 \\ 0
\end{pmatrix}$
and 
$\displaystyle
\begin{pmatrix}
	0 \\ 3
\end{pmatrix}$
is 
\begin{equation}
\begin{pmatrix}
	\displaystyle \frac{-1}{2} &
	\displaystyle \frac{1}{3}
\end{pmatrix}
	{\mathbf{x}} = 1
\end{equation}


%---------------------------------------------------------
\end{document}
