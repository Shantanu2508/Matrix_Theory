%
%--------------------------------------------------------
%  Assignment 1
%  Shantanu Yadav, EE20MTech12001
%  IIT Hyderabad
%--------------------------------------------------------
%
\documentclass[12pt]{article}
\usepackage{amsmath}
\usepackage{graphicx}
\textheight=9.0in
\textwidth=6.0in
\newcommand{\myvec}[1]{\ensuremath{\begin{pmatrix}#1\end{pmatrix}}}
\thispagestyle{empty}

\begin{document}
\begin{center}
	{\Large \bf IIT Hyderabad} \\ \vspace{2ex}
	{\large \bf SHANTANU YADAV, EE20MTECH12001 }\\
	\vspace{2ex}
	{\large \bf ASSIGNMENT 1} \\
\end{center}
	\hrule

\vspace{2ex}
\begin{center}
{\underline{\Large \bf Lines and Planes}}
\end{center}

\section*{Problem Statement}
Find the equations of the lines which intercepts on the both the axes and whose sum and product are 1 and -6 respectively.

\section*{Solution}
The equation of line in terms of vector notations can be written as
\begin{align}
	{\mathbf{n}^T}{\mathbf x} = b     \qquad \text{ where } \qquad 
	\mathbf{n} = \myvec{n_{11} \\  n_{12}}, 
\end{align}
	or
\begin{align}
	\myvec{n_{11} &  n_{12}} \mathbf{x} = b	\label{eq2}
\end{align}
Let the intercepts be $\myvec{a \\ 0}$ and $\myvec{0 \\ b}$, respectively.

\begin{figure}[htbp]
\centering
\includegraphics[width=0.7\linewidth]{stline_plot_new.png}
\caption{}
\end{figure}

\noindent
Given that: \qquad \quad $ a + b = 1 $, \qquad and \qquad \quad $ ab = -6$ 

\begin{equation*}
\implies b = \frac{-6}{a} \  
\implies a^2 - a -6 =0
\implies (a-3)(a+2)=0
\end{equation*}
\begin{equation}
	\implies (a,b)=(3,-2) \text{ and } (-2,3)
\end{equation}

\noindent
When the line passes through $\myvec{3 \\ 0}$ and $\myvec{0 \\ -2}$, 
respectively,
we get, upon substitution in (\ref{eq2}):
	\[ 3 n_{11} = b \qquad \implies \qquad n_{11} = \frac{b}{3} \]
	\[-2 n_{12} = b \qquad \implies \qquad n_{12} =-\frac{b}{2} \]
Therefore, the equation of first line is
\begin{align*}
	\myvec{\frac{b}{3} & \frac{-b}{2} } {\mathbf{x}} = b
\end{align*}
%
$\implies$
\begin{align}
	\myvec{ \frac{1}{3} & \frac{-1}{2} } {\mathbf{x}} = 1
\end{align}
Similarly, the equation of second line, which passes through 
$\myvec{ -2 \\ 0}$ and $\myvec{ 0 \\ 3}$ 
is 
\begin{align}
	\myvec{ \frac{-1}{2} & \frac{1}{3} } {\mathbf{x}} = 1
\end{align}

%---------------------------------------------------------
\end{document}
