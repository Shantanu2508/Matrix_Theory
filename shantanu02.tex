%
%--------------------------------------------------------
%  Assignment 1
%  Shantanu Yadav, EE20MTech12001
%  IIT Hyderabad
%--------------------------------------------------------
%
%\documentclass[12pt]{IEEEtran}
\documentclass[12pt]{article}
\usepackage{amsmath}
\usepackage{graphicx}
\usepackage{tabularx}
\usepackage{listings}
\lstset{frame=single,breaklines=true,columns=fullflexible}

\textheight=9.5in
\textwidth=6.0in
\topmargin=-0.5in
\newcommand{\myvec}[1]{\ensuremath{\begin{pmatrix}#1\end{pmatrix}}}
\thispagestyle{empty}

\begin{document}
\begin{center}
	{\large \bf EE5609 ASSIGNMENT 1} \\
	\vspace{2ex}
	{\large \bf SHANTANU YADAV, EE20MTECH12001 }\\
	{\Large \bf IIT Hyderabad} \\ \vspace{2ex}
\end{center}
	\hrule

\vspace{2ex}
\begin{center}
{\underline{\Large \bf Lines and Planes}}
\end{center}

%---------------------------------------------------------
\noindent 
The python solution code is available at
\begin{lstlisting}
	https://github.com/Shantanu2508/Matrix_Theory/blob/master/stline.py
\end{lstlisting}
%---------------------------------------------------------

\section*{Problem Statement}
Find the equations of the lines which intercepts on the both the axes and whose sum and product are 1 and -6 respectively.

\section*{Solution}
The equation of line in terms of vector notations can be written as
\begin{align}
	{\mathbf{n}^T}{\mathbf x} = b     \qquad \text{ where } \qquad 
	\mathbf{n} = \myvec{n_{11} \\  n_{12}}, 
\end{align}
	or
\begin{align}
	\myvec{n_{11} &  n_{12}} \mathbf{x} = b	\label{eq2}
\end{align}
Let the intercepts be $\myvec{a \\ 0}$ and $\myvec{0 \\ b}$, respectively.

\begin{figure}[htbp]
\centering
\includegraphics[width=0.7\linewidth]{stline_plot.png}
\caption{}
\end{figure}

\noindent
Given that: \qquad \quad $ a + b = 1 $, \qquad and \qquad \quad $ ab = -6$ \\
The quadratic equation whose roots are the x and y intercepts can be written 
as :
\begin{equation*}
	x^2 - (\text{sum of roots})x + (\text{product of roots}) = 0
\end{equation*}
\begin{equation*}
	\implies x^2 - x -6 =0
\end{equation*}
%\begin{equation*}
%\implies b = \frac{-6}{a} \  
%\implies a^2 - a -6 =0
%\implies (a-3)(a+2)=0
%\end{equation*}
\begin{equation}
	\implies x=(3,-2) \text{ and corresponding y intercepts are  } (-2,3)
\end{equation}

\noindent
When the line passes through $\myvec{3 \\ 0}$ and $\myvec{0 \\ -2}$, 
respectively,
we get, upon substitution in (\ref{eq2}):
	\[ 3 n_{11} = b \qquad \implies \qquad n_{11} = \frac{b}{3} \]
	\[-2 n_{12} = b \qquad \implies \qquad n_{12} =-\frac{b}{2} \]
Therefore, the equation of first line is
\begin{align*}
	\myvec{\frac{b}{3} & \frac{-b}{2} } {\mathbf{x}} = b
\end{align*}
%
$\implies$
\begin{align}
	\myvec{ \frac{1}{3} & \frac{-1}{2} } {\mathbf{x}} = 1
\end{align}
Similarly, the equation of second line, which passes through 
$\myvec{ -2 \\ 0}$ and $\myvec{ 0 \\ 3}$ 
is 
\begin{align}
	\myvec{ \frac{-1}{2} & \frac{1}{3} } {\mathbf{x}} = 1
\end{align}

\begin{table}[htbp]
	\centering
	\begin{tabularx}{0.7\linewidth}{|c|c|X|X|} \hline 
	$x-$intercept & $y-$intercept & $n_{11}$ & $n_{12}$ \\ \hline \hline
		3	&	-2	&	 1/3	&	-1/2 \\ \hline
  	       -2	&	 3	&	-1/2	&	 1/3 \\ \hline
	\end{tabularx}
	\caption{}
\end{table}
\end{document}
