\documentclass[journal,12pt,twocolumn]{IEEEtran}

\usepackage{setspace}
\usepackage{gensymb}

\singlespacing


\usepackage[cmex10]{amsmath}

\usepackage{amsthm}

\usepackage{mathrsfs}
\usepackage{txfonts}
\usepackage{stfloats}
\usepackage{bm}
\usepackage{cite}
\usepackage{cases}
\usepackage{subfig}

\usepackage{longtable}
\usepackage{multirow}

\usepackage{enumitem}
\usepackage{mathtools}
%\usepackage{steinmetz}
\usepackage{tikz}
\usepackage{circuitikz}
\usepackage{verbatim}
%\usepackage{tfrupee}
\usepackage[breaklinks=true]{hyperref}

\usepackage{tkz-euclide}

\usetikzlibrary{calc,math}
\usepackage{listings}
    \usepackage{color}                                            %%
    \usepackage{array}                                            %%
    \usepackage{longtable}                                        %%
    \usepackage{calc}                                             %%
    \usepackage{multirow}                                         %%
    \usepackage{hhline}                                           %%
    \usepackage{ifthen}                                           %%
    \usepackage{lscape}     
\usepackage{multicol}
\usepackage{chngcntr}
%\usepackage{enumerate,float}
\DeclareMathOperator*{\Res}{Res}

\renewcommand\thesection{\arabic{section}}
\renewcommand\thesubsection{\thesection.\arabic{subsection}}
\renewcommand\thesubsubsection{\thesubsection.\arabic{subsubsection}}

\renewcommand\thesectiondis{\arabic{section}}
\renewcommand\thesubsectiondis{\thesectiondis.\arabic{subsection}}
\renewcommand\thesubsubsectiondis{\thesubsectiondis.\arabic{subsubsection}}


\hyphenation{op-tical net-works semi-conduc-tor}
\def\inputGnumericTable{}                                 %%

\lstset{
%language=C,
frame=single, 
breaklines=true,
columns=fullflexible
}
\begin{document}


\newtheorem{theorem}{Theorem}[section]
\newtheorem{problem}{Problem}
\newtheorem{proposition}{Proposition}[section]
\newtheorem{lemma}{Lemma}[section]
\newtheorem{corollary}[theorem]{Corollary}
\newtheorem{example}{Example}[section]
\newtheorem{definition}[problem]{Definition}

\newcommand{\BEQA}{\begin{eqnarray}}
\newcommand{\EEQA}{\end{eqnarray}}
\newcommand{\define}{\stackrel{\triangle}{=}}
\bibliographystyle{IEEEtran}
\providecommand{\mbf}{\mathbf}
\providecommand{\pr}[1]{\ensuremath{\Pr\left(#1\right)}}
\providecommand{\qfunc}[1]{\ensuremath{Q\left(#1\right)}}
\providecommand{\sbrak}[1]{\ensuremath{{}\left[#1\right]}}
\providecommand{\lsbrak}[1]{\ensuremath{{}\left[#1\right.}}
\providecommand{\rsbrak}[1]{\ensuremath{{}\left.#1\right]}}
\providecommand{\brak}[1]{\ensuremath{\left(#1\right)}}
\providecommand{\lbrak}[1]{\ensuremath{\left(#1\right.}}
\providecommand{\rbrak}[1]{\ensuremath{\left.#1\right)}}
\providecommand{\cbrak}[1]{\ensuremath{\left\{#1\right\}}}
\providecommand{\lcbrak}[1]{\ensuremath{\left\{#1\right.}}
\providecommand{\rcbrak}[1]{\ensuremath{\left.#1\right\}}}
\theoremstyle{remark}
\newtheorem{rem}{Remark}
\newcommand{\sgn}{\mathop{\mathrm{sgn}}}
\providecommand{\abs}[1]{\left\vert#1\right\vert}
\providecommand{\res}[1]{\Res\displaylimits_{#1}} 
\providecommand{\norm}[1]{\left\lVert#1\right\rVert}
%\providecommand{\norm}[1]{\lVert#1\rVert}
\providecommand{\mtx}[1]{\mathbf{#1}}
\providecommand{\mean}[1]{E\left[ #1 \right]}
\providecommand{\fourier}{\overset{\mathcal{F}}{ \rightleftharpoons}}
%\providecommand{\hilbert}{\overset{\mathcal{H}}{ \rightleftharpoons}}
\providecommand{\system}{\overset{\mathcal{H}}{ \longleftrightarrow}}
	%\newcommand{\solution}[2]{\textbf{Solution:}{#1}}
\newcommand{\solution}{\noindent \textbf{Solution: }}
\newcommand{\cosec}{\,\text{cosec}\,}
\providecommand{\dec}[2]{\ensuremath{\overset{#1}{\underset{#2}{\gtrless}}}}
\newcommand{\myvec}[1]{\ensuremath{\begin{pmatrix}#1\end{pmatrix}}}
\newcommand{\mydet}[1]{\ensuremath{\begin{vmatrix}#1\end{vmatrix}}}
\numberwithin{equation}{subsection}
\makeatletter
\@addtoreset{figure}{problem}
\makeatother
\let\StandardTheFigure\thefigure
\let\vec\mathbf
\renewcommand{\thefigure}{\theproblem}
\def\putbox#1#2#3{\makebox[0in][l]{\makebox[#1][l]{}\raisebox{\baselineskip}[0in][0in]{\raisebox{#2}[0in][0in]{#3}}}}
     \def\rightbox#1{\makebox[0in][r]{#1}}
     \def\centbox#1{\makebox[0in]{#1}}
     \def\topbox#1{\raisebox{-\baselineskip}[0in][0in]{#1}}
     \def\midbox#1{\raisebox{-0.5\baselineskip}[0in][0in]{#1}}
\vspace{3cm}
\title{EE5609 Assignment 13}
\author{SHANTANU YADAV, EE20MTECH12001 }
\maketitle
\newpage
\bigskip
\renewcommand{\thefigure}{\theenumi}
\renewcommand{\thetable}{\theenumi}

\section{Problem}
Let $\vec{F}$ be a field and let $\vec{V}$ be the space of polynomial functions $f$ from $\vec{F}$
into $\vec{F}$, given by
\begin{align*}
	f(x)=c_0+c_1x+\cdots +c_nx^n
\end{align*}
Let $\vec{D}$ be a linear differentiation transformation defined as
\begin{align*}
	(\vec{D}f)(x)=\frac{df(x)}{dx}
\end{align*}
Then find the range and null space of 
$\vec{D}$.\\
Let $\vec{R}$ be the field of real numbers and let $\vec{V}$ be the space of all functions from 
$\vec{R}$ into $\vec{R}$ which are continous. Let $\vec{T}$ be linear transformation defined by
\begin{align*}
	(\vec{T}f)(x)=\int_{0}^{x} f(t)\,dt
\end{align*}
Find the range and null space of $\vec{T}$.
\section{Explanation}
Let the vector space $\vec{V}$ be defined as 
\begin{align}
	\vec{V}=\cbrak{f:\vec{F}\rightarrow \vec{F} : f(x)=\sum_{k=0}^{n}c_kx^k, \ c_k \in \vec{F} }
\end{align}
Differentiation transformation is defined as a function which maps the vectors in $\vec{F}$ into $\vec{F}$ such that
\begin{align}
	(\vec{D}f)(x)=\frac{df(x)}{dx}\\
	\implies 
	\vec{D}f=\sum_{k=0}^{n}kc_kx^{k-1}=g(x)
\end{align}
Since $g(x) \in \vec{V}$ therefore the transformation  $\vec{D}$ is defined from $\vec{V}$ into $\vec{V}$. Thus the range of $\vec{D}$ is the entire vector space $\vec{V}$. Now consider the nullspace 
for differentiation transformation defined as
\begin{align}
	\vec{N}=\cbrak{f \in \vec{V} : \vec{D}f=0} \\
	\vec{D}f=0 \implies f=c
\end{align}
where c is a constant. Such a polynomial is known as constant polynomial. Therefore
\begin{align}
	\vec{N}=\cbrak{f=c : f \in \vec{V}, c \in \vec{F} \ \text{where c is a constant}}
\end{align}
Now consider the vector space defined as 
\begin{align}
	\vec{V}=\cbrak{f:\vec{R} \rightarrow \vec{R} : \text{f is continous}}
\end{align}
Integration transformation is defined as 
\begin{align}
	(\vec{T}f)(x)=\int_{0}^{x}f(t)\,dt
\end{align}
Let
\begin{align}
	F(x)=\int_{0}^{x}f(t)\,dt
\end{align}
Since f is continous function we have $\left\vert f(t) \right\vert \leq M$ $\forall \ t \in [0,x] $
and $\left\vert M \right\vert \geq 0$, it follows that
\begin{align}
	\left\vert F(x+h) - F(x) \right\vert = 
	\left\vert \int_{0}^{h} f(t)\,dt \right\vert \leq M\left\vert h \right\vert
\end{align}
which shows that F(x) is also continous and thus
\begin{align}
	F(x) \in \vec{V}
\end{align}
Therefore the transformation $\vec{T}$ is defined from $\vec{V}$ into $\vec{V}$. 
Thus the range of $\vec{T}$ is the entire vector space $\vec{V}$. Now consider the nullspace for 
intergration transformation defined as
\begin{align}
        \vec{N}=\cbrak{f \in \vec{V} : \vec{T}f=0} \\
        \vec{T}f=0 \implies \int_{0}^{x}f(t)\,dt=0  \\
	\implies f(t)=0
\end{align}
Therefore nullspace for integration transformation is
\begin{align}
	\vec{N}=\cbrak{0}
\end{align}
\end{document}
