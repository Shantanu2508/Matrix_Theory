\documentclass[journal,12pt,twocolumn]{IEEEtran}

\usepackage{setspace}
\usepackage{gensymb}

\singlespacing


\usepackage[cmex10]{amsmath}

\usepackage{amsthm}

\usepackage{mathrsfs}
\usepackage{txfonts}
\usepackage{stfloats}
\usepackage{bm}
\usepackage{cite}
\usepackage{cases}
\usepackage{subfig}

\usepackage{longtable}
\usepackage{multirow}

\usepackage{enumitem}
\usepackage{mathtools}
%\usepackage{steinmetz}
\usepackage{tikz}
\usepackage{circuitikz}
\usepackage{verbatim}
%\usepackage{tfrupee}
\usepackage[breaklinks=true]{hyperref}

\usepackage{tkz-euclide}

\usetikzlibrary{calc,math}
\usepackage{listings}
    \usepackage{color}                                            %%
    \usepackage{array}                                            %%
    \usepackage{longtable}                                        %%
    \usepackage{calc}                                             %%
    \usepackage{multirow}                                         %%
    \usepackage{hhline}                                           %%
    \usepackage{ifthen}                                           %%
    \usepackage{lscape}     
\usepackage{multicol}
\usepackage{chngcntr}
%\usepackage{enumerate,float}
\DeclareMathOperator*{\Res}{Res}

\renewcommand\thesection{\arabic{section}}
\renewcommand\thesubsection{\thesection.\arabic{subsection}}
\renewcommand\thesubsubsection{\thesubsection.\arabic{subsubsection}}

\renewcommand\thesectiondis{\arabic{section}}
\renewcommand\thesubsectiondis{\thesectiondis.\arabic{subsection}}
\renewcommand\thesubsubsectiondis{\thesubsectiondis.\arabic{subsubsection}}


\hyphenation{op-tical net-works semi-conduc-tor}
\def\inputGnumericTable{}                                 %%

\lstset{
%language=C,
frame=single, 
breaklines=true,
columns=fullflexible
}
\begin{document}


\newtheorem{theorem}{Theorem}[section]
\newtheorem{problem}{Problem}
\newtheorem{proposition}{Proposition}[section]
\newtheorem{lemma}{Lemma}[section]
\newtheorem{corollary}[theorem]{Corollary}
\newtheorem{example}{Example}[section]
\newtheorem{definition}[problem]{Definition}

\newcommand{\BEQA}{\begin{eqnarray}}
\newcommand{\EEQA}{\end{eqnarray}}
\newcommand{\define}{\stackrel{\triangle}{=}}
\bibliographystyle{IEEEtran}
\providecommand{\mbf}{\mathbf}
\providecommand{\pr}[1]{\ensuremath{\Pr\left(#1\right)}}
\providecommand{\qfunc}[1]{\ensuremath{Q\left(#1\right)}}
\providecommand{\sbrak}[1]{\ensuremath{{}\left[#1\right]}}
\providecommand{\lsbrak}[1]{\ensuremath{{}\left[#1\right.}}
\providecommand{\rsbrak}[1]{\ensuremath{{}\left.#1\right]}}
\providecommand{\brak}[1]{\ensuremath{\left(#1\right)}}
\providecommand{\lbrak}[1]{\ensuremath{\left(#1\right.}}
\providecommand{\rbrak}[1]{\ensuremath{\left.#1\right)}}
\providecommand{\cbrak}[1]{\ensuremath{\left\{#1\right\}}}
\providecommand{\lcbrak}[1]{\ensuremath{\left\{#1\right.}}
\providecommand{\rcbrak}[1]{\ensuremath{\left.#1\right\}}}
\theoremstyle{remark}
\newtheorem{rem}{Remark}
\newcommand{\sgn}{\mathop{\mathrm{sgn}}}
\providecommand{\abs}[1]{\left\vert#1\right\vert}
\providecommand{\res}[1]{\Res\displaylimits_{#1}} 
\providecommand{\norm}[1]{\left\lVert#1\right\rVert}
%\providecommand{\norm}[1]{\lVert#1\rVert}
\providecommand{\mtx}[1]{\mathbf{#1}}
\providecommand{\mean}[1]{E\left[ #1 \right]}
\providecommand{\fourier}{\overset{\mathcal{F}}{ \rightleftharpoons}}
%\providecommand{\hilbert}{\overset{\mathcal{H}}{ \rightleftharpoons}}
\providecommand{\system}{\overset{\mathcal{H}}{ \longleftrightarrow}}
	%\newcommand{\solution}[2]{\textbf{Solution:}{#1}}
\newcommand{\solution}{\noindent \textbf{Solution: }}
\newcommand{\cosec}{\,\text{cosec}\,}
\providecommand{\dec}[2]{\ensuremath{\overset{#1}{\underset{#2}{\gtrless}}}}
\newcommand{\myvec}[1]{\ensuremath{\begin{pmatrix}#1\end{pmatrix}}}
\newcommand{\mydet}[1]{\ensuremath{\begin{vmatrix}#1\end{vmatrix}}}
\numberwithin{equation}{subsection}
\makeatletter
\@addtoreset{figure}{problem}
\makeatother
\let\StandardTheFigure\thefigure
\let\vec\mathbf
\renewcommand{\thefigure}{\theproblem}
\def\putbox#1#2#3{\makebox[0in][l]{\makebox[#1][l]{}\raisebox{\baselineskip}[0in][0in]{\raisebox{#2}[0in][0in]{#3}}}}
     \def\rightbox#1{\makebox[0in][r]{#1}}
     \def\centbox#1{\makebox[0in]{#1}}
     \def\topbox#1{\raisebox{-\baselineskip}[0in][0in]{#1}}
     \def\midbox#1{\raisebox{-0.5\baselineskip}[0in][0in]{#1}}
\vspace{3cm}
\title{EE5609 Assignment 18}
\author{SHANTANU YADAV, EE20MTECH12001 }
\maketitle
\newpage
\bigskip
\renewcommand{\thefigure}{\theenumi}
%\renewcommand{\thetable}{\theenumi}

\section{Problem}
Let $\vec{M}$ be a $n\times n$ Hermitian matrix of rank $k,k\neq n$. If $\lambda\neq 0$ is
an eigenvalue of $\vec{M}$ with corresponding unit column vector $\vec{u}$,
with $\vec{Mu}=\lambda\vec{u}$ then which of the following are true?
\begin{enumerate}
	\item $rank\brak{\vec{M}-\lambda\vec{uu^*}}=k-1$
	\item $rank\brak{\vec{M}-\lambda\vec{uu^*}}=k$
	\item $rank\brak{\vec{M}-\lambda\vec{uu^*}}=k+1$
	\item $\brak{\vec{M}-\lambda\vec{uu^*}}^n=\vec{M}^n-\lambda^n\vec{uu^*}$
\end{enumerate}

\section{Explanation}

\begin{table}[htbp]
        \centering
	\begin{tabular}{|m{2.0in}|m{5.0in}|} \hline
		\textbf{Objective} & \textbf{Explanation} \\ \hline
	Rank of $\vec{M}-\lambda \vec{uu^*}$ & Since 
	\begin{align}
	rank\brak{\vec{A}-\vec{B}} \geq rank\brak{\vec{A}}-rank\brak{\vec{B}} \\
		\implies rank\brak{\vec{M}-\lambda\vec{uu^*}} \geq 
	rank\brak{\vec{M}}-rank\brak{\vec{uu^*}} \\
		\implies rank\brak{\vec{M}-\lambda\vec{uu^*}} \geq k-rank\brak{\vec{uu^*}}
	\end{align} 
If $\vec{A}$ is a non-zero column vector of order $m\times 1$ and $\vec{B}$ is a non-zero row vector 
of order $1\times n$ then $rank\brak{AB}=1$. So,
	\begin{align}
	rank\brak{\vec{uu^*}}=1 \\
	\implies rank\brak{\vec{M}-\lambda\vec{uu^*}} \geq k-1 \label{eq1}
	\end{align}
Also since,
	\begin{align}
	\vec{M}-\lambda\vec{uu^*} = \vec{M}-\vec{Muu^*}=\vec{M}\brak{I-\vec{uu^*}}
	\end{align}
and
	\begin{align}
	rank\brak{\vec{M}\brak{\vec{I}-\vec{uu^*}}}
	\leq min\brak{rank\brak{\vec{M}},rank\brak{\vec{I}-\vec{uu^*}}} \\
	\implies
	rank\brak{\vec{M}\brak{\vec{I}-\vec{uu^*}}} \leq k \label{eq2}
	\end{align}
Thus we have from (\ref{eq1}) and (\ref{eq2}) that
	\begin{align}
	rank\brak{\vec{M}-\lambda\vec{uu^*}} = k-1 \ \text{or} \ k
	\end{align}
	Consider a matrix 
	\begin{align}
		\vec{M}=\myvec{1&0\\0&0}
	\end{align}
\end{tabular}
\end{table}

\clearpage
\begin{table}[htbp]
        \centering
\begin{tabular}{|m{2.0in}|m{5.0in}|} \hline
		\textbf{Objective} & \textbf{Explanation} \\ \hline
		&
	such that $rank\brak{M}=1$. The eigenvalue of $\vec{M}$ is $\lambda=1$ 
	and the corresponding eigenvector is
	\begin{align}
	\vec{u}=\myvec{1\\0}
	\end{align}
	Thus we have,
	\begin{align}
	\vec{M}-\lambda\vec{uu^*}=\myvec{1&0\\0&0}-\myvec{1\\0}\myvec{1&0} \\
		=\myvec{1&0\\0&0}-\myvec{1&0\\0&0} \\
		=\myvec{0&0\\0&0} \\
		\implies rank\brak{\vec{M}-\lambda\vec{uu^*}}=0
	\end{align}
Hence if $rank\brak{\vec{M}}=k$ 
		then $rank\brak{\vec{M}-\lambda\vec{uu^*}}=k-1$. \\ 
		& \\ \hline
	$\brak{\vec{M}-\lambda\vec{uu^*}}^n=\vec{M}^n-\lambda^n\vec{uu^*}$  & 
		Let the given statement be 
	P(n):$\brak{\vec{M}-\lambda\vec{uu^*}}^n=\vec{M}^n-\lambda^n\vec{uu^*}$.
		It can be seen that P(1) is true. Assume P(n) is true for some 
		$k\in \vec{N}$ such that
	\begin{align}
	\brak{\vec{M}-\lambda\vec{uu^*}}^k=\vec{M}^k-\lambda^k\vec{uu^*}
	\end{align}
	Now to prove that P(k+1) is true we have
	\begin{align}
	\brak{\vec{M}-\lambda\vec{uu^*}}^{k+1}
	=\brak{\vec{M}-\lambda\vec{uu^*}}\brak{\vec{M}-\lambda\vec{uu^*}}^k \\
	=\brak{\vec{M}-\lambda\vec{uu^*}}\brak{\vec{M}^k-\lambda^k\vec{uu^*}} \\
	=\vec{M}^{k+1}-\lambda^k\vec{Muu^*}-\lambda\vec{M}^k\vec{uu^*} + 
		\lambda^{k+1}\vec{uu^*uu^*} \\
	=\vec{M}^{k+1}-\lambda^{k+1}\vec{uu^*}-\lambda^{k+1}\vec{uu^*} +
		\lambda^{k+1}\vec{u}\norm{\vec{u}}^2\vec{u}^* \\
	=\vec{M}^{k+1}-2\lambda^{k+1}\vec{uu^*}+\lambda^{k+1}\vec{uu^*} \\
	=\vec{M}^{k+1}-\lambda^{k+1}\vec{uu^*}
	\end{align}
	Hence, by the Principle of Mathematical Induction P(n) is true for all 
	$n$.\\ \hline
		Answer& (1) and (4) \\ \hline
\end{tabular}
        \caption{} \label{1}
\end{table}
\end{document}
