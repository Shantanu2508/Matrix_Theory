\documentclass[journal,12pt,twocolumn]{IEEEtran}

\usepackage{setspace}
\usepackage{gensymb}

\singlespacing


\usepackage[cmex10]{amsmath}

\usepackage{amsthm}

\usepackage{mathrsfs}
\usepackage{txfonts}
\usepackage{stfloats}
\usepackage{bm}
\usepackage{cite}
\usepackage{cases}
\usepackage{subfig}

\usepackage{longtable}
\usepackage{multirow}

\usepackage{enumitem}
\usepackage{mathtools}
%\usepackage{steinmetz}
\usepackage{tikz}
\usepackage{circuitikz}
\usepackage{verbatim}
%\usepackage{tfrupee}
\usepackage[breaklinks=true]{hyperref}

\usepackage{tkz-euclide}

\usetikzlibrary{calc,math}
\usepackage{listings}
    \usepackage{color}                                            %%
    \usepackage{array}                                            %%
    \usepackage{longtable}                                        %%
    \usepackage{calc}                                             %%
    \usepackage{multirow}                                         %%
    \usepackage{hhline}                                           %%
    \usepackage{ifthen}                                           %%
    \usepackage{lscape}     
\usepackage{multicol}
\usepackage{chngcntr}
%\usepackage{enumerate,float}
\DeclareMathOperator*{\Res}{Res}

\renewcommand\thesection{\arabic{section}}
\renewcommand\thesubsection{\thesection.\arabic{subsection}}
\renewcommand\thesubsubsection{\thesubsection.\arabic{subsubsection}}

\renewcommand\thesectiondis{\arabic{section}}
\renewcommand\thesubsectiondis{\thesectiondis.\arabic{subsection}}
\renewcommand\thesubsubsectiondis{\thesubsectiondis.\arabic{subsubsection}}


\hyphenation{op-tical net-works semi-conduc-tor}
\def\inputGnumericTable{}                                 %%

\lstset{
%language=C,
frame=single, 
breaklines=true,
columns=fullflexible
}
\begin{document}


\newtheorem{theorem}{Theorem}[section]
\newtheorem{problem}{Problem}
\newtheorem{proposition}{Proposition}[section]
\newtheorem{lemma}{Lemma}[section]
\newtheorem{corollary}[theorem]{Corollary}
\newtheorem{example}{Example}[section]
\newtheorem{definition}[problem]{Definition}

\newcommand{\BEQA}{\begin{eqnarray}}
\newcommand{\EEQA}{\end{eqnarray}}
\newcommand{\define}{\stackrel{\triangle}{=}}
\bibliographystyle{IEEEtran}
\providecommand{\mbf}{\mathbf}
\providecommand{\pr}[1]{\ensuremath{\Pr\left(#1\right)}}
\providecommand{\qfunc}[1]{\ensuremath{Q\left(#1\right)}}
\providecommand{\sbrak}[1]{\ensuremath{{}\left[#1\right]}}
\providecommand{\lsbrak}[1]{\ensuremath{{}\left[#1\right.}}
\providecommand{\rsbrak}[1]{\ensuremath{{}\left.#1\right]}}
\providecommand{\brak}[1]{\ensuremath{\left(#1\right)}}
\providecommand{\lbrak}[1]{\ensuremath{\left(#1\right.}}
\providecommand{\rbrak}[1]{\ensuremath{\left.#1\right)}}
\providecommand{\cbrak}[1]{\ensuremath{\left\{#1\right\}}}
\providecommand{\lcbrak}[1]{\ensuremath{\left\{#1\right.}}
\providecommand{\rcbrak}[1]{\ensuremath{\left.#1\right\}}}
\theoremstyle{remark}
\newtheorem{rem}{Remark}
\newcommand{\sgn}{\mathop{\mathrm{sgn}}}
\providecommand{\abs}[1]{\left\vert#1\right\vert}
\providecommand{\res}[1]{\Res\displaylimits_{#1}} 
\providecommand{\norm}[1]{\left\lVert#1\right\rVert}
%\providecommand{\norm}[1]{\lVert#1\rVert}
\providecommand{\mtx}[1]{\mathbf{#1}}
\providecommand{\mean}[1]{E\left[ #1 \right]}
\providecommand{\fourier}{\overset{\mathcal{F}}{ \rightleftharpoons}}
%\providecommand{\hilbert}{\overset{\mathcal{H}}{ \rightleftharpoons}}
\providecommand{\system}{\overset{\mathcal{H}}{ \longleftrightarrow}}
	%\newcommand{\solution}[2]{\textbf{Solution:}{#1}}
\newcommand{\solution}{\noindent \textbf{Solution: }}
\newcommand{\cosec}{\,\text{cosec}\,}
\providecommand{\dec}[2]{\ensuremath{\overset{#1}{\underset{#2}{\gtrless}}}}
\newcommand{\myvec}[1]{\ensuremath{\begin{pmatrix}#1\end{pmatrix}}}
\newcommand{\mydet}[1]{\ensuremath{\begin{vmatrix}#1\end{vmatrix}}}
\numberwithin{equation}{subsection}
\makeatletter
\@addtoreset{figure}{problem}
\makeatother
\let\StandardTheFigure\thefigure
\let\vec\mathbf
\renewcommand{\thefigure}{\theproblem}
\def\putbox#1#2#3{\makebox[0in][l]{\makebox[#1][l]{}\raisebox{\baselineskip}[0in][0in]{\raisebox{#2}[0in][0in]{#3}}}}
     \def\rightbox#1{\makebox[0in][r]{#1}}
     \def\centbox#1{\makebox[0in]{#1}}
     \def\topbox#1{\raisebox{-\baselineskip}[0in][0in]{#1}}
     \def\midbox#1{\raisebox{-0.5\baselineskip}[0in][0in]{#1}}
\vspace{3cm}
\title{EE5609 Assignment 12}
\author{SHANTANU YADAV, EE20MTECH12001 }
\maketitle
\newpage
\bigskip
\renewcommand{\thefigure}{\theenumi}
\renewcommand{\thetable}{\theenumi}

\section{Problem}
Let $\vec{W}$ be the subspace of $\vec{C}^3$ spanned by 
$\alpha_1=\myvec{1\\0\\i}$ and $\alpha_2=\myvec{1\\i\\1+i}$.
\begin{enumerate}[label=\emph{\alph*)}]
	\item Show that $\alpha_1$ and $\alpha_2$ form a basis for $\vec{W}$.\\
	\item Show that the vectors $\beta_1=\myvec{1\\1\\0}$ and 
		$\beta_2=\myvec{1\\i\\1+i}$ are in $\vec{W}$ and form another
		basis for $\vec{W}$. \\
	\item What are the coordinates of $\alpha_1$ and $\alpha_2$ in the
		ordered basis $\cbrak{\beta_1, \beta_2} $ for $\vec{W}$.
\end{enumerate}
\section{Explanation}
\begin{enumerate}[label=\emph{\alph*)}]
\item 
It is given that $\alpha_1$ and $\alpha_2$ span $\vec{W}$. For $\alpha_1$ and
$\alpha_2$ to be the basis for $\vec{W}$ they must be linearly independent.
Let
\begin{align}
	S_1=\cbrak{\alpha_1,\alpha_2}=\cbrak{\myvec{1\\0\\i},\myvec{1+i\\1\\-1}}
\end{align}
Using row reduction on matrix $\vec{A}=\myvec{\alpha_1 & \alpha_2}$
\begin{align}
	\myvec{1 & 1+i \\ 0 & 1 \\ i & -1}
	\xleftrightarrow[]{R_3 \leftarrow R_3-iR_1}
	\myvec{1 & 1+i \\ 0 & 1 \\ 0 & -i}   \label{rrefA}
\end{align}
Since $\vec{A}$ is a full-rank matrix the column vectors are linearly 
independent. Therefore $S_1= \cbrak{ \alpha_1, \alpha_2 } $ is a basis set for 
$\vec{W}$.
\item
\begin{align}
	\beta_1=\myvec{1\\1\\0} \\
	\beta_2=\myvec{1\\i\\1+i} 
\end{align}
		Since column vectors of $\vec{A}=\myvec{\alpha_1 & \alpha_2}$ are basis for $\vec{W}$ and if $\beta_1$ and $\beta_2 \ \in \vec{W}$ there exist a unique solution $\vec{x}$ such that
\begin{align}
	\myvec{\alpha_1 & \alpha_2}\vec{x}=\beta_1 
\end{align}
Using row reduction on augmented matrix
		\begin{align}
			\myvec{1 & 1+i & | & 1 \\
			       0 & 1   & | & 1 \\
			       i & -1  & | & 0}\\
			\xleftrightarrow[]{R3 \leftarrow R_3-iR-1}
			\myvec{1 & 1+i & | & 1 \\
			       0 & 1   & | & 1 \\
			       0 & -i  & | & -i}\\
			\xleftrightarrow[]{R_3 \leftarrow R_3-R_2}
			\myvec{1 & 1+i & | & 1 \\
			       0 & 1   & | & 1 \\
			       0 & 0  &  | & 0}\\
			\xleftrightarrow[]{R_1 \leftarrow R_1-(i+1)R_2}
			\myvec{1 & 0 &  | & -i \\
			       0 & 1   &| & 1 \\
			       0 & 0  & | & 0 }\\
			       \implies 
			       \vec{x}=\myvec{-i\\1}
		\end{align}
Therefore $\beta_1  \in \ \vec{W}$.\\
		Similarly for $\beta_2 \in \ \vec{W}$ there must exist a unique solution $\vec{x}$
		such that
		\begin{align}
			\myvec{\alpha_2 & \alpha_2}\vec{x}=\beta_2
		\end{align}
		Using row reduction on augmented matrix 
		\begin{align}
			\myvec{1 & 1+i & | & 1 \\
                               0 & 1   & | & i \\
                               i & -1  & | & 1+i}\\
                        \xleftrightarrow[]{R3 \leftarrow R_3-iR-1}
                        \myvec{1 & 1+i & | & 1 \\
                               0 & 1   & | & i \\
                               0 & -i  & | & 1}\\
                        \xleftrightarrow[]{R_3 \leftarrow R_3+iR_2}
                        \myvec{1 & 1+i & | & 1 \\
                               0 & 1   & | & i \\
                               0 & 0  &  | & 0}\\
                        \xleftrightarrow[]{R_1 \leftarrow R_1-(i+1)R_2}
                        \myvec{1 & 0 &  | & 2-i \\
                               0 & 1   &| & i \\
                               0 & 0  & | & 0 }
		\end{align}
		\begin{align}
                               \implies
                               \vec{x}=\myvec{2-i\\i}
		\end{align}
Therefore $\beta_2 \in \vec{W}$.\\
Consider 
\begin{align}
	S_2=\cbrak{\beta_1,\beta_2}=\cbrak{\myvec{1\\1\\0},\myvec{1\\i\\1+i}}
\end{align}
and also let 
\begin{align}
	\vec{B}=\myvec{1 & 1 \\
		       1 & i \\
		       0 & 1+i}
\end{align}
Using row reduction on matrix $\vec{B}$
\begin{align}
	\myvec{1 & 1 \\
               1 & i \\
               0 & 1+i}
        \xleftrightarrow[]{R_2 \leftarrow R_2-R_1}
	\myvec{1 & 1 \\
               0 & i-1 \\
	       0 & 1+i}    \label{Binv}
\end{align}
Since $\vec{B}$ is a full rank matrix the column vectors are linearly independent.\\
Let $\alpha$ be any vector in the subspace $\vec{W}$, then it can be expressed as 
span $\cbrak{\alpha_1,\alpha_2}$ i.e
\begin{align}
	\alpha = \myvec{\alpha_1 & \alpha_2}\vec{x_1} =\vec{A}\vec{x_1} \label{Ax}
\end{align}
$S_2=\cbrak{\beta_1,\beta_2}$ spans $\vec{W}$ if any vector $\alpha \in \ \vec{W}$ can be 
expressed as
\begin{align}
	\alpha=\myvec{\beta_1,\beta_2}\vec{x_2}=\vec{B}\vec{x_2} \label{Bx}
\end{align}
From (\ref{Ax}) and (\ref{Bx}) we conclude
\begin{align}
	\vec{B}\vec{x_2}=\vec{A}\vec{x_1}\\
	\implies \vec{x_2}=\vec{B}^{-1}\vec{A}\vec{x_1} \label{x2}
\end{align}
Therefore from (\ref{x2}) $\vec{x_2}$ exists if $\vec{B}$ is invertible. From
(\ref{Binv}) we conclude $\vec{x_2}$ exists and hence any vector $\alpha \in \ \vec{W}$
can be expressed as span$\cbrak{\beta_1,\beta_2}$. Therefore $\cbrak{\beta_1,\beta_2}$
is basis for $\vec{W}$.
\item
	Since $\alpha_1,\alpha_2 \in \ \vec{W}$ and $\cbrak{\beta_1,\beta_2}$ are ordered
	basis for $\vec{W}$ there must exist unique value of $\vec{x}$ such that
	\begin{align}
		\myvec{\beta_1 & \beta_2}\vec{x}=\alpha_1 \label{a1}\\
		\myvec{\beta_1 & \beta_2}\vec{x}=\alpha_2 \label{a2}
	\end{align}
Using row reduction on (\ref{a1}) we get,
\begin{align}
	\myvec{1 & 1 & | & 1 \\
	       1 & i & | & 0 \\
	       0 & 1+i &| & i}\\
	\xleftrightarrow[]{R_2 \leftarrow R_2-R_1}
	\myvec{1 & 1 & | & 1 \\
               0 & i-1 & | & -1 \\
               0 & 1+i &| & i}\\
        \xleftrightarrow[]{R_2 \leftarrow \frac{R_2}{i-1}}
	\myvec{1 & 1 & | & 1 \\
	       0 & 1 & | & \frac{1+i}{2} \\
               0 & 1+i &| & i}\\
	\xleftrightarrow[]{R_3 \leftarrow R_3 - (i+1)R_2}
	\myvec{1 & 1 & | & 1 \\
               0 & 1 & | & \frac{1+i}{2} \\
               0 & 0 &| & 0}\\
	\xleftrightarrow[]{R_1 \leftarrow R_2- R_1}
	\myvec{1 & 0 & | & \frac{1-i}{2} \\
               0 & 1 & | & \frac{1+i}{2} \\
               0 & 0 &| & 0}\\
	\implies 
	\vec{x}=\frac{1}{2}\myvec{1-i \\ 1+i}
\end{align}
and now applying row reduction on (\ref{a2}) we get,
\begin{align}
        \myvec{1 & 1 & | & 1+i \\
	       1 & i & | & 1 \\
               0 & 1+i &| & -1}\\
        \xleftrightarrow[]{R_2 \leftarrow R_2-R_1}
        \myvec{1 & 1 & | & 1+i \\
               0 & i-1 & | & -i \\
               0 & 1+i &| & -1}\\
        \xleftrightarrow[]{R_2 \leftarrow \frac{R_2}{i-1}}
        \myvec{1 & 1 & | & 1+i \\
               0 & 1 & | & \frac{-1+i}{2} \\
               0 & 1+i &| & -1}\\
        \xleftrightarrow[]{R_3 \leftarrow R_3 - (i+1)R_2}
        \myvec{1 & 1 & | & 1+i \\
               0 & 1 & | & \frac{-1+i}{2} \\
               0 & 0 &| & 0}\\
        \xleftrightarrow[]{R_1 \leftarrow R_2-R_1}
        \myvec{1 & 0 & | & \frac{3+i}{2} \\
               0 & 1 & | & \frac{-1+i}{2} \\
               0 & 0 &| & 0}\\
        \implies
        \vec{x}=\frac{1}{2}\myvec{3+i \\ -1+i}
\end{align}
\end{enumerate}
\end{document}
