\documentclass[journal,12pt,twocolumn]{IEEEtran}

\usepackage{setspace}
\usepackage{gensymb}

\singlespacing


\usepackage[cmex10]{amsmath}

\usepackage{amsthm}

\usepackage{mathrsfs}
\usepackage{txfonts}
\usepackage{stfloats}
\usepackage{bm}
\usepackage{cite}
\usepackage{cases}
\usepackage{subfig}

\usepackage{longtable}
\usepackage{multirow}

\usepackage{enumitem}
\usepackage{mathtools}
%\usepackage{steinmetz}
\usepackage{tikz}
\usepackage{circuitikz}
\usepackage{verbatim}
%\usepackage{tfrupee}
\usepackage[breaklinks=true]{hyperref}

\usepackage{tkz-euclide}

\usetikzlibrary{calc,math}
\usepackage{listings}
    \usepackage{color}                                            %%
    \usepackage{array}                                            %%
    \usepackage{longtable}                                        %%
    \usepackage{calc}                                             %%
    \usepackage{multirow}                                         %%
    \usepackage{hhline}                                           %%
    \usepackage{ifthen}                                           %%
    \usepackage{lscape}     
\usepackage{multicol}
\usepackage{chngcntr}
%\usepackage{enumerate,float}
\DeclareMathOperator*{\Res}{Res}

\renewcommand\thesection{\arabic{section}}
\renewcommand\thesubsection{\thesection.\arabic{subsection}}
\renewcommand\thesubsubsection{\thesubsection.\arabic{subsubsection}}

\renewcommand\thesectiondis{\arabic{section}}
\renewcommand\thesubsectiondis{\thesectiondis.\arabic{subsection}}
\renewcommand\thesubsubsectiondis{\thesubsectiondis.\arabic{subsubsection}}


\hyphenation{op-tical net-works semi-conduc-tor}
\def\inputGnumericTable{}                                 %%

\lstset{
%language=C,
frame=single, 
breaklines=true,
columns=fullflexible
}
\begin{document}


\newtheorem{theorem}{Theorem}[section]
\newtheorem{problem}{Problem}
\newtheorem{proposition}{Proposition}[section]
\newtheorem{lemma}{Lemma}[section]
\newtheorem{corollary}[theorem]{Corollary}
\newtheorem{example}{Example}[section]
\newtheorem{definition}[problem]{Definition}

\newcommand{\BEQA}{\begin{eqnarray}}
\newcommand{\EEQA}{\end{eqnarray}}
\newcommand{\define}{\stackrel{\triangle}{=}}
\bibliographystyle{IEEEtran}
\providecommand{\mbf}{\mathbf}
\providecommand{\pr}[1]{\ensuremath{\Pr\left(#1\right)}}
\providecommand{\qfunc}[1]{\ensuremath{Q\left(#1\right)}}
\providecommand{\sbrak}[1]{\ensuremath{{}\left[#1\right]}}
\providecommand{\lsbrak}[1]{\ensuremath{{}\left[#1\right.}}
\providecommand{\rsbrak}[1]{\ensuremath{{}\left.#1\right]}}
\providecommand{\brak}[1]{\ensuremath{\left(#1\right)}}
\providecommand{\lbrak}[1]{\ensuremath{\left(#1\right.}}
\providecommand{\rbrak}[1]{\ensuremath{\left.#1\right)}}
\providecommand{\cbrak}[1]{\ensuremath{\left\{#1\right\}}}
\providecommand{\lcbrak}[1]{\ensuremath{\left\{#1\right.}}
\providecommand{\rcbrak}[1]{\ensuremath{\left.#1\right\}}}
\theoremstyle{remark}
\newtheorem{rem}{Remark}
\newcommand{\sgn}{\mathop{\mathrm{sgn}}}
\providecommand{\abs}[1]{\left\vert#1\right\vert}
\providecommand{\res}[1]{\Res\displaylimits_{#1}} 
\providecommand{\norm}[1]{\left\lVert#1\right\rVert}
%\providecommand{\norm}[1]{\lVert#1\rVert}
\providecommand{\mtx}[1]{\mathbf{#1}}
\providecommand{\mean}[1]{E\left[ #1 \right]}
\providecommand{\fourier}{\overset{\mathcal{F}}{ \rightleftharpoons}}
%\providecommand{\hilbert}{\overset{\mathcal{H}}{ \rightleftharpoons}}
\providecommand{\system}{\overset{\mathcal{H}}{ \longleftrightarrow}}
	%\newcommand{\solution}[2]{\textbf{Solution:}{#1}}
\newcommand{\solution}{\noindent \textbf{Solution: }}
\newcommand{\cosec}{\,\text{cosec}\,}
\providecommand{\dec}[2]{\ensuremath{\overset{#1}{\underset{#2}{\gtrless}}}}
\newcommand{\myvec}[1]{\ensuremath{\begin{pmatrix}#1\end{pmatrix}}}
\newcommand{\mydet}[1]{\ensuremath{\begin{vmatrix}#1\end{vmatrix}}}
\numberwithin{equation}{subsection}
\makeatletter
\@addtoreset{figure}{problem}
\makeatother
\let\StandardTheFigure\thefigure
\let\vec\mathbf
\renewcommand{\thefigure}{\theproblem}
\def\putbox#1#2#3{\makebox[0in][l]{\makebox[#1][l]{}\raisebox{\baselineskip}[0in][0in]{\raisebox{#2}[0in][0in]{#3}}}}
     \def\rightbox#1{\makebox[0in][r]{#1}}
     \def\centbox#1{\makebox[0in]{#1}}
     \def\topbox#1{\raisebox{-\baselineskip}[0in][0in]{#1}}
     \def\midbox#1{\raisebox{-0.5\baselineskip}[0in][0in]{#1}}
\vspace{3cm}
\title{EE5609 Assignment 15}
\author{SHANTANU YADAV, EE20MTECH12001 }
\maketitle
\newpage
\bigskip
\renewcommand{\thefigure}{\theenumi}
\renewcommand{\thetable}{\theenumi}

\section{Problem}
The linear operator $\vec{T}$ on $\vec{R}^2$ defined by 
\begin{align*}
	\vec{T}\myvec{x_1\\x_2}=\myvec{x_1\\0}
\end{align*}
is represented by the matrix
\begin{align*}
	\vec{A}=\myvec{1&0\\
		       0&0}
\end{align*}
Prove that if $\vec{S}$ is a linear operator on $\vec{R}^2$ such that $\vec{S}^2=\vec{S}$, 
then $\vec{S}=\vec{0}$, or $\vec{S}=\vec{I}$, or there is an ordered basis $\vec{B}$ for $\vec{R}^2$ 
such that $[\vec{S}]_B=\vec{A}$.
\section{Explanation}
Let us consider a matrix $\vec{S}$ formed from the linear combination of columns of standard 
ordred basis matrix $\vec{I}$ of $\vec{R}^2$ where 
\begin{align}
	\vec{I}=\myvec{1&0\\0&1}
\end{align}
and therefore,
\begin{align}
	\vec{S}&=\alpha\myvec{1\\0}+\beta\myvec{0\\1}\\
	       &=\myvec{\alpha&0\\
			0&\beta}
\end{align}
Then,
\begin{align}
	\vec{S}^2&=\myvec{\alpha&0\\
			 0&\beta}\myvec{\alpha&0\\
			 		0&\beta}\\
		 &=\myvec{\alpha^2&0\\
		 	    0&\beta^2}
\end{align}
From the question if $\vec{S}^2=\vec{S}$ then,
\begin{align}
	&\implies\vec{S}^2-\vec{S}=\vec{0} \\
	&\implies\myvec{\alpha^2&0\\0&\beta^2}-\myvec{\alpha&0\\0&\beta}=\vec{0}\\
	&\implies\myvec{\alpha(\alpha-1)&0\\0&\beta(\beta-1)}=\vec{0}
\end{align}
Thus the solution set is
\begin{align}
	\myvec{\alpha\\\beta}=\cbrak{\myvec{0\\0},\myvec{1\\0},\myvec{0\\1},\myvec{1\\1}}
\end{align}
For $\myvec{\alpha\\\beta}=\myvec{0\\0}$ we get,
\begin{align}
	\vec{S}=\myvec{0&0\\0&0}=\vec{0}
\end{align}
For $\myvec{\alpha\\\beta}=\myvec{1\\1}$ we get,
\begin{align}
	\vec{S}=\myvec{1&0\\0&1}=\vec{I}
\end{align}
For $\myvec{\alpha\\\beta}=\myvec{1\\0}$ we get,
\begin{align}
        \vec{S}=\myvec{1&0\\0&0}=\vec{A}
\end{align}
\end{document}
