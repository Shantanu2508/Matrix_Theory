\documentclass[journal,12pt,twocolumn]{IEEEtran}

\usepackage{setspace}
\usepackage{gensymb}

\singlespacing


\usepackage[cmex10]{amsmath}

\usepackage{amsthm}

\usepackage{mathrsfs}
\usepackage{txfonts}
\usepackage{stfloats}
\usepackage{bm}
\usepackage{cite}
\usepackage{cases}
\usepackage{subfig}

\usepackage{longtable}
\usepackage{multirow}

\usepackage{enumitem}
\usepackage{mathtools}
%\usepackage{steinmetz}
\usepackage{tikz}
\usepackage{circuitikz}
\usepackage{verbatim}
%\usepackage{tfrupee}
\usepackage[breaklinks=true]{hyperref}

\usepackage{tkz-euclide}

\usetikzlibrary{calc,math}
\usepackage{listings}
    \usepackage{color}                                            %%
    \usepackage{array}                                            %%
    \usepackage{longtable}                                        %%
    \usepackage{calc}                                             %%
    \usepackage{multirow}                                         %%
    \usepackage{hhline}                                           %%
    \usepackage{ifthen}                                           %%
    \usepackage{lscape}     
\usepackage{multicol}
\usepackage{chngcntr}

\DeclareMathOperator*{\Res}{Res}

\renewcommand\thesection{\arabic{section}}
\renewcommand\thesubsection{\thesection.\arabic{subsection}}
\renewcommand\thesubsubsection{\thesubsection.\arabic{subsubsection}}

\renewcommand\thesectiondis{\arabic{section}}
\renewcommand\thesubsectiondis{\thesectiondis.\arabic{subsection}}
\renewcommand\thesubsubsectiondis{\thesubsectiondis.\arabic{subsubsection}}


\hyphenation{op-tical net-works semi-conduc-tor}
\def\inputGnumericTable{}                                 %%

\lstset{
%language=C,
frame=single, 
breaklines=true,
columns=fullflexible
}
\begin{document}


\newtheorem{theorem}{Theorem}[section]
\newtheorem{problem}{Problem}
\newtheorem{proposition}{Proposition}[section]
\newtheorem{lemma}{Lemma}[section]
\newtheorem{corollary}[theorem]{Corollary}
\newtheorem{example}{Example}[section]
\newtheorem{definition}[problem]{Definition}

\newcommand{\BEQA}{\begin{eqnarray}}
\newcommand{\EEQA}{\end{eqnarray}}
\newcommand{\define}{\stackrel{\triangle}{=}}
\bibliographystyle{IEEEtran}
\providecommand{\mbf}{\mathbf}
\providecommand{\pr}[1]{\ensuremath{\Pr\left(#1\right)}}
\providecommand{\qfunc}[1]{\ensuremath{Q\left(#1\right)}}
\providecommand{\sbrak}[1]{\ensuremath{{}\left[#1\right]}}
\providecommand{\lsbrak}[1]{\ensuremath{{}\left[#1\right.}}
\providecommand{\rsbrak}[1]{\ensuremath{{}\left.#1\right]}}
\providecommand{\brak}[1]{\ensuremath{\left(#1\right)}}
\providecommand{\lbrak}[1]{\ensuremath{\left(#1\right.}}
\providecommand{\rbrak}[1]{\ensuremath{\left.#1\right)}}
\providecommand{\cbrak}[1]{\ensuremath{\left\{#1\right\}}}
\providecommand{\lcbrak}[1]{\ensuremath{\left\{#1\right.}}
\providecommand{\rcbrak}[1]{\ensuremath{\left.#1\right\}}}
\theoremstyle{remark}
\newtheorem{rem}{Remark}
\newcommand{\sgn}{\mathop{\mathrm{sgn}}}
\providecommand{\abs}[1]{\left\vert#1\right\vert}
\providecommand{\res}[1]{\Res\displaylimits_{#1}} 
\providecommand{\norm}[1]{\left\lVert#1\right\rVert}
%\providecommand{\norm}[1]{\lVert#1\rVert}
\providecommand{\mtx}[1]{\mathbf{#1}}
\providecommand{\mean}[1]{E\left[ #1 \right]}
\providecommand{\fourier}{\overset{\mathcal{F}}{ \rightleftharpoons}}
%\providecommand{\hilbert}{\overset{\mathcal{H}}{ \rightleftharpoons}}
\providecommand{\system}{\overset{\mathcal{H}}{ \longleftrightarrow}}
	%\newcommand{\solution}[2]{\textbf{Solution:}{#1}}
\newcommand{\solution}{\noindent \textbf{Solution: }}
\newcommand{\cosec}{\,\text{cosec}\,}
\providecommand{\dec}[2]{\ensuremath{\overset{#1}{\underset{#2}{\gtrless}}}}
\newcommand{\myvec}[1]{\ensuremath{\begin{pmatrix}#1\end{pmatrix}}}
\newcommand{\mydet}[1]{\ensuremath{\begin{vmatrix}#1\end{vmatrix}}}
\numberwithin{equation}{subsection}
\makeatletter
\@addtoreset{figure}{problem}
\makeatother
\let\StandardTheFigure\thefigure
\let\vec\mathbf
\renewcommand{\thefigure}{\theproblem}
\def\putbox#1#2#3{\makebox[0in][l]{\makebox[#1][l]{}\raisebox{\baselineskip}[0in][0in]{\raisebox{#2}[0in][0in]{#3}}}}
     \def\rightbox#1{\makebox[0in][r]{#1}}
     \def\centbox#1{\makebox[0in]{#1}}
     \def\topbox#1{\raisebox{-\baselineskip}[0in][0in]{#1}}
     \def\midbox#1{\raisebox{-0.5\baselineskip}[0in][0in]{#1}}
\vspace{3cm}
\title{EE5609 Assignment 7}
\author{SHANTANU YADAV, EE20MTECH12001 }
\maketitle
\newpage
\bigskip
\renewcommand{\thefigure}{\theenumi}
\renewcommand{\thetable}{\theenumi}

The python solution code is available at
\begin{lstlisting}
https://github.com/Shantanu2508/Matrix_Theory/blob/master/Assignment_6/qr.py
\end{lstlisting}

\section{Problem}
Perform QR decomposition on matrix A given by
\begin{align*}
	A = \myvec{3 & -4 \\ -4 & 3} 
\end{align*}
\section{Explanation}
Representing matrix A in terms of its column vectors as
\begin{align}
	A=\myvec{\vec{a} & \vec{b}}
\end{align}
Let
\begin{align}
	\vec{q_1}=\frac{\vec{a}}{\norm{\vec{a}}} \label{q1} 
\end{align}
An orthonormal vector to $\vec{q_1}$ can be obtained by subtracting the projection of $\vec{b}$ on $\vec{q_1}$ from $\vec{b}$. Thus
\begin{align}
	\vec{q_2}=\frac{\vec{b}-k\vec{q_1}}{\norm{\vec{b}-k\vec{q_1}}} \label{q2}
\end{align}
where
\begin{align}
	k=\frac{\vec{b}^T\vec{q_1}}{\norm{\vec{q_1}}^2} \label{k}
\end{align}
From (\ref{q1}) and (\ref{q2})
\begin{align}
	&\vec{a}=\norm{\vec{a}}\vec{q_1} \\
	&\vec{b}=k\vec{q_1} + \norm{\vec{b}-k\vec{q_1}}\vec{q_2} \\
	\implies 
	&\myvec{\vec{a} & \vec{b}}=\myvec{\vec{q_1} & \vec{q_2}}
        \myvec{\norm{\vec{a}} & k \\ 0 & \norm{\vec{b}-k\vec{q_1}}} \\
	&\implies \vec{A}=\vec{Q}\vec{R}
\end{align}
QR decomposition of a matrix A is essentially representation of column vectors of matrix A in terms of linear combination of orthonormal basis of column space of A.
\section{Solution}
For matrix A
\begin{align}
	\vec{a}=\myvec{3\\-4}, \ \vec{b}=\myvec{-4\\3} \\
\end{align}
Let
\begin{align}
	\vec{q_1}=\frac{1}{\sqrt{5}}\myvec{3\\-4} \\
\end{align}
From (\ref{q2}) and (\ref{k})
\begin{align}
	\vec{q_2}=\frac{1}{\sqrt{5}}\myvec{-4\\3}\\
	\implies 
	\vec{Q}=\frac{1}{\sqrt{5}}\myvec{3 & -4 \\ -4 & 3} \\
	\vec{R}=\myvec{\sqrt{5} & 0 \\ 0 & \sqrt{5}}
\end{align}
Therefore the matrix A can be decomposed as 
\begin{align}
	\vec{A}=
	\myvec{\frac{3}{\sqrt{5}} & -\frac{4}{\sqrt{5}} \\
	      -\frac{4}{\sqrt{5}} & \frac{3}{\sqrt{5}} }
	\myvec{\sqrt{5} & 0 \\ 0 & \sqrt{5}}
\end{align}
\end{document}
