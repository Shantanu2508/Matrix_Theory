\documentclass[journal,12pt,twocolumn]{IEEEtran}

\usepackage{setspace}
\usepackage{gensymb}

\singlespacing


\usepackage[cmex10]{amsmath}

\usepackage{amsthm}

\usepackage{mathrsfs}
\usepackage{txfonts}
\usepackage{stfloats}
\usepackage{bm}
\usepackage{cite}
\usepackage{cases}
\usepackage{subfig}

\usepackage{longtable}
\usepackage{multirow}

\usepackage{enumitem}
\usepackage{mathtools}
%\usepackage{steinmetz}
\usepackage{tikz}
\usepackage{circuitikz}
\usepackage{verbatim}
%\usepackage{tfrupee}
\usepackage[breaklinks=true]{hyperref}

\usepackage{tkz-euclide}

\usetikzlibrary{calc,math}
\usepackage{listings}
    \usepackage{color}                                            %%
    \usepackage{array}                                            %%
    \usepackage{longtable}                                        %%
    \usepackage{calc}                                             %%
    \usepackage{multirow}                                         %%
    \usepackage{hhline}                                           %%
    \usepackage{ifthen}                                           %%
    \usepackage{lscape}     
\usepackage{multicol}
\usepackage{chngcntr}

\DeclareMathOperator*{\Res}{Res}

\renewcommand\thesection{\arabic{section}}
\renewcommand\thesubsection{\thesection.\arabic{subsection}}
\renewcommand\thesubsubsection{\thesubsection.\arabic{subsubsection}}

\renewcommand\thesectiondis{\arabic{section}}
\renewcommand\thesubsectiondis{\thesectiondis.\arabic{subsection}}
\renewcommand\thesubsubsectiondis{\thesubsectiondis.\arabic{subsubsection}}


\hyphenation{op-tical net-works semi-conduc-tor}
\def\inputGnumericTable{}                                 %%

\lstset{
%language=C,
frame=single, 
breaklines=true,
columns=fullflexible
}
\begin{document}


\newtheorem{theorem}{Theorem}[section]
\newtheorem{problem}{Problem}
\newtheorem{proposition}{Proposition}[section]
\newtheorem{lemma}{Lemma}[section]
\newtheorem{corollary}[theorem]{Corollary}
\newtheorem{example}{Example}[section]
\newtheorem{definition}[problem]{Definition}

\newcommand{\BEQA}{\begin{eqnarray}}
\newcommand{\EEQA}{\end{eqnarray}}
\newcommand{\define}{\stackrel{\triangle}{=}}
\bibliographystyle{IEEEtran}
\providecommand{\mbf}{\mathbf}
\providecommand{\pr}[1]{\ensuremath{\Pr\left(#1\right)}}
\providecommand{\qfunc}[1]{\ensuremath{Q\left(#1\right)}}
\providecommand{\sbrak}[1]{\ensuremath{{}\left[#1\right]}}
\providecommand{\lsbrak}[1]{\ensuremath{{}\left[#1\right.}}
\providecommand{\rsbrak}[1]{\ensuremath{{}\left.#1\right]}}
\providecommand{\brak}[1]{\ensuremath{\left(#1\right)}}
\providecommand{\lbrak}[1]{\ensuremath{\left(#1\right.}}
\providecommand{\rbrak}[1]{\ensuremath{\left.#1\right)}}
\providecommand{\cbrak}[1]{\ensuremath{\left\{#1\right\}}}
\providecommand{\lcbrak}[1]{\ensuremath{\left\{#1\right.}}
\providecommand{\rcbrak}[1]{\ensuremath{\left.#1\right\}}}
\theoremstyle{remark}
\newtheorem{rem}{Remark}
\newcommand{\sgn}{\mathop{\mathrm{sgn}}}
\providecommand{\abs}[1]{\left\vert#1\right\vert}
\providecommand{\res}[1]{\Res\displaylimits_{#1}} 
\providecommand{\norm}[1]{\left\lVert#1\right\rVert}
%\providecommand{\norm}[1]{\lVert#1\rVert}
\providecommand{\mtx}[1]{\mathbf{#1}}
\providecommand{\mean}[1]{E\left[ #1 \right]}
\providecommand{\fourier}{\overset{\mathcal{F}}{ \rightleftharpoons}}
%\providecommand{\hilbert}{\overset{\mathcal{H}}{ \rightleftharpoons}}
\providecommand{\system}{\overset{\mathcal{H}}{ \longleftrightarrow}}
	%\newcommand{\solution}[2]{\textbf{Solution:}{#1}}
\newcommand{\solution}{\noindent \textbf{Solution: }}
\newcommand{\cosec}{\,\text{cosec}\,}
\providecommand{\dec}[2]{\ensuremath{\overset{#1}{\underset{#2}{\gtrless}}}}
\newcommand{\myvec}[1]{\ensuremath{\begin{pmatrix}#1\end{pmatrix}}}
\newcommand{\mydet}[1]{\ensuremath{\begin{vmatrix}#1\end{vmatrix}}}
\numberwithin{equation}{subsection}
\makeatletter
\@addtoreset{figure}{problem}
\makeatother
\let\StandardTheFigure\thefigure
\let\vec\mathbf
\renewcommand{\thefigure}{\theproblem}
\def\putbox#1#2#3{\makebox[0in][l]{\makebox[#1][l]{}\raisebox{\baselineskip}[0in][0in]{\raisebox{#2}[0in][0in]{#3}}}}
     \def\rightbox#1{\makebox[0in][r]{#1}}
     \def\centbox#1{\makebox[0in]{#1}}
     \def\topbox#1{\raisebox{-\baselineskip}[0in][0in]{#1}}
     \def\midbox#1{\raisebox{-0.5\baselineskip}[0in][0in]{#1}}
\vspace{3cm}
\title{EE5609 Assignment 8}
\author{SHANTANU YADAV, EE20MTECH12001 }
\maketitle
\newpage
\bigskip
\renewcommand{\thefigure}{\theenumi}
\renewcommand{\thetable}{\theenumi}

The python solution code is available at
\begin{lstlisting}
https://github.com/Shantanu2508/Matrix_Theory/blob/master/Assignment_8/assignment8.py
\end{lstlisting}

\section{Problem}
Find the foot of perpendicular from point \mbox{$B=\myvec{3\\ -2\\ 0}$} to the 
plane $\myvec{2 & 3 & -4}\vec{x} = -5$.

\section{Solution}
Let us consider orthogonal vectors $\vec{m}_1$ and $\vec{m}_2$ to the given
normal vector $\vec{n}$. Let $\vec{m} = \myvec{a\\ b\\ c}$.

Then, 
\begin{align}
	\vec{m}^T\vec{n} = 0 \\
	\implies \quad \myvec{a & b & c}\myvec{2\\ 3\\-4} = 0   \label{mTn} \\
	\implies \quad 2a + 3b -4c = 0 
\end{align}
Let $a=1$, \ $b=0$, so that
\begin{align}
	\vec{m}_1 = \myvec{1 \\ 0 \\ \frac{1}{2}} 
\end{align}
and $a=0$, \ $b=1$, so that
\begin{align}
	\vec{m}_2 = \myvec{0 \\ 1 \\ \frac{3}{4}} 
\end{align}
We, now, solve the equation
\begin{align}
	\vec{M}\vec{x} = \vec{b} 	\label{matrixeq}
\end{align}
which, upon substitution, becomes
\begin{align}
\myvec{1 & 0\\ 0 & 1 \\ \frac{1}{2} & \frac{3}{4}}\vec{x}=\myvec{3 \\ -2 \\ 0} 
\end{align}
	Any $m\times n$ matrix $\vec{M}$ can be factorized in SVD form as
\begin{align}
	\vec{M} = \vec{U}\vec{S}\vec{V}^T	\label{svd}
\end{align}
where $\vec{U}$ and $\vec{V}$ are matrices of eigen vectors which are orthogonal. Columns of $\vec{V}$ are the eigen vectors of $\vec{M}^T\vec{M}$,columns of $\vec{U}$ are the eigen vectors of $\vec{M}\vec{M}^T$ and $\vec{S}$ is the diagonal matrix of singular values of $\vec{M}$ of the eigenvalues of $\vec{M}^T\vec{M}$.
\begin{align}
	\vec{M}^T\vec{M} = 
	\myvec{\frac{10}{8} & \frac{3}{8} \\ \frac{3}{8} & \frac{25}{16} }
\end{align}
Putting (\ref{svd}) into (\ref{matrixeq}), we get
\begin{align}
	\vec{U}\vec{S} \vec{V}^T\vec{x} = \vec{b} \\
    \implies \qquad \vec{x} = \vec{V}\vec{S}_{+}\vec{U}^T\vec{b} \label{calcx}
\end{align}
where \ $\vec{S}_{+}$ is the Moore-Penrose Pseudoinverse of $\vec{S}$.

The eigenvalues of $\vec{M}^T\vec{M}$:
\begin{align}
	\mydet{\vec{M}^T\vec{M}-\lambda \vec{I}} = 0 \\
	\implies \quad
	\mydet{\frac{10}{8}-\lambda & \frac{3}{8} \\ \frac{3}{8} & \frac{25}{16}-\lambda } = 0 \\
	\implies \quad
	\lambda^2 - \frac{45}{16}\lambda + \frac{116}{64} = 0
\end{align}
	So, the eigenvalues are 
\begin{align}
	\lambda_1 = \frac{29}{16}	\\
	\lambda_2 = 1
\end{align}
For $\lambda_1 = \frac{29}{16}$, the eigen vector $\vec{v_1}$ can be calculated using row reduction as :
\begin{align}
	\myvec{-\frac{9}{16} & \frac{3}{8} \\ \frac{3}{8} & -\frac{4}{16}} 
	\xleftrightarrow[]{R_1 \leftarrow - \frac{16}{9}R_1}
	\myvec{1 & -\frac{2}{3} \\ \frac{3}{8} & -\frac{4}{16}} \\
	\xleftrightarrow[]{R_2 \leftarrow R_2-R_1}
	\myvec{1 & -\frac{2}{3} \\ 0 & 0} 
\end{align}
Hence, 
\begin{align}
	\vec{v_1}= \myvec{\frac{2}{\sqrt{13}} \\ \frac{3}{\sqrt{13}}}
\end{align}
Similarly, for $\lambda_2=1$, 
\begin{align}
	\vec{v_2}= \myvec{-\frac{3}{\sqrt{13}} \\ \frac{2}{\sqrt{13}}}
\end{align}
Thus, 
\begin{align}
	\vec{V}= \myvec{\frac{2}{\sqrt{13}} & -\frac{3}{\sqrt{13}} \\
		\frac{3}{\sqrt{13}} & \frac{2}{\sqrt{13}} }
\end{align}
Now, 
\begin{align}
	\vec{M}\vec{M}^T = 
	\myvec{1 & 0 & \frac{1}{2} \\ 0 & 1 & \frac{3}{4} \\
		\frac{1}{2} & \frac{3}{4} & \frac{13}{16} }
\end{align}
Now, calculating eigenvalues of $\vec{M}\vec{M}^T$
\begin{align}
	\mydet{1-\lambda & 0 & \frac{1}{2} \\ 0 & 1-\lambda & \frac{3}{4} \\
		\frac{1}{2} & \frac{3}{4} & \frac{13}{16}-\lambda } = 0  
\end{align}
	So, the eigenvalues are 
\begin{align}
	\lambda_1 = \frac{29}{16}	\\
	\lambda_2 = 1	\\
	\lambda_3 = 0
\end{align}
For $\lambda_1 = \frac{29}{16}$, the eigen vector can be computed as:
\begin{align}
	\myvec{1-\frac{29}{16} & 0 & \frac{1}{2} \\
	 0 & 1 - \frac{29}{16} & \frac{3}{4} \\
	\frac{1}{2} & \frac{3}{4} & \frac{13}{16}-\frac{29}{16} }  \\
	\xleftrightarrow[]{}
	\myvec{-\frac{13}{16} & 0 & \frac{1}{2}\\ 
	0 & -\frac{13}{16} & \frac{3}{4} \\
	\frac{1}{2} & \frac{3}{4} & -1} \\
	\xleftrightarrow[]{R_1 \leftarrow - \frac{16}{13}R_1}
	\myvec{1 & 0 & -\frac{8}{3} \\ 0 & -\frac{13}{16} & \frac{3}{4} \\
	\frac{1}{2} & \frac{3}{4} & -1} \\
	\xleftrightarrow[]{R_3 \leftarrow R_3-\frac{1}{2}R_1}
	\myvec{1 & 0 & -\frac{8}{3} \\ 0 & -\frac{13}{16} & \frac{3}{4} \\
	0 & \frac{3}{4} & -\frac{9}{13} } \\
	\xleftrightarrow[]{R_2 \leftarrow -\frac{16}{13}R_2}
	\myvec{1 & 0 & -\frac{8}{3} \\ 0 & 1 & -\frac{12}{13} \\
	0 & \frac{3}{4} & -\frac{9}{13} } 
\end{align}
\begin{align}
	\xleftrightarrow[]{R_2 \leftarrow R_3-\frac{3}{4}R_2}
	\myvec{1 & 0 & -\frac{8}{3} \\ 0 & 1 & -\frac{12}{13} \\
	0 & 0 & 0 }
\end{align}
Hence, the eigen vector $\vec{u}_1$:
\begin{align}
	\vec{u_1}= \myvec{\frac{8}{\sqrt{377}} \\ \frac{12}{\sqrt{377}} \\
	\frac{13}{\sqrt{377}} }
\end{align}
For $\lambda_2=1$, the eigen vector is:
\begin{align}
	\myvec{1-1 & 0 & \frac{1}{2} \\ 0 & 1-1 & \frac{3}{4} \\
	\frac{1}{2} & \frac{3}{4} & \frac{13}{16}-1 }  \\
	\xleftrightarrow[]{}
	\myvec{0 & 0 & \frac{1}{2} \\ 0 & 0 & \frac{3}{4} \\
	\frac{1}{2} & \frac{3}{4} & -\frac{3}{16} }  
\end{align}
Hence, the eigen vector $\vec{u}_2$:
\begin{align}
	\vec{u_2}= \myvec{\frac{3}{\sqrt{13}} \\ -\frac{2}{\sqrt{13}} \\ 0 }
\end{align}
Similarly, for $\lambda_3=0$, the eigen vector is:
\begin{align}
	\myvec{1 & 0 & \frac{1}{2} \\ 0 & 1 & \frac{3}{4} \\
	\frac{1}{2} & \frac{3}{4} & \frac{13}{16} }  \\
	\xleftrightarrow[]{R_3 \leftarrow R_3-\frac{1}{2}R_1-\frac{3}{4}R_2}
	\myvec{0 & 0 & \frac{1}{2} \\ 0 & 0 & \frac{3}{4} \\
	0	& 	0	&	0 } 
\end{align}
Hence, the eigen vector $\vec{u}_3$:
\begin{align}
	\vec{u_3}= \myvec{\frac{2}{\sqrt{29}} \\ \frac{3}{\sqrt{29}} \\ 
	-\frac{4}{\sqrt{29}} }
\end{align}
So, the orthonormal matrix $\vec{U}$ of eigen vectors is:
\begin{align}
   \vec{U}= \myvec{
	   \frac{8}{\sqrt{377}} & \frac{3}{\sqrt{13}} & \frac{2}{\sqrt{29}}  \\ 
	   \frac{12}{\sqrt{377}} & -\frac{2}{\sqrt{13}} & \frac{3}{\sqrt{29}}\\ 
	   \frac{13}{\sqrt{377}} & 0 & -\frac{4}{\sqrt{29}} }
\end{align}
The matrix of singular values of $\vec{M}$ is: 
\begin{align}
   \vec{S}= \myvec{ \frac{\sqrt{29}}{4} & 0 & 0 \\
	   0 & 1 & 0 \\ 0 & 0 & 0}
\end{align}
The Moore-Penrose pseudoinverse of $\vec{S}$ is computed as
\begin{align}
	\vec{S}_{+} &= (\vec{S}\vec{S}^T)^{-1}\vec{S}^T 	\\
		&= \myvec{ \frac{4}{\sqrt{29}} & 0 & 0 \\ 0 & 1 & 0 }
\end{align}
To solve for $\vec{x}$ in (\ref{calcx}), noting that $\vec{b}=\myvec{3 \\ -2 \\ 0}$, 
\begin{align}
\vec{U}^T\vec{b} = \myvec{
	0 \\ \sqrt{13} \\ 0 } \\
\vec{S}_{+}\vec{U}^T\vec{b} = \myvec{ 0 \\ \sqrt{13} }
\end{align}
Thus, the foot of perpendicular is:
\begin{align}
\vec{x} = \vec{V}\vec{S}_{+}\vec{U}^T\vec{b} 
	&= \myvec{\frac{2}{\sqrt{13}} & -\frac{3}{\sqrt{13}} \\
		\frac{3}{\sqrt{13}} & \frac{2}{\sqrt{13}} }
	  \myvec{ 0 \\ \sqrt{13} }
	\\
	\implies \quad
\vec{x} &= \myvec{ -3 \\ 2 }  	\label{xeq}
\end{align}

The solution can be verified using
\begin{align}
	\vec{M}^T\vec{M}\vec{x} = \vec{M}^T\vec{b}
\end{align}
The LHS gives
\begin{align}
	& \vec{M}^T\vec{M}\vec{x} 
	= \myvec{\frac{10}{8} & \frac{3}{8} \\ \frac{3}{8} & \frac{25}{16} }
	\myvec{-3 \\ 2} \\
	\implies
	& \vec{M}^T\vec{M}\vec{x} = \myvec{-3 \\ 2} 
\end{align}
Now, finding $\vec{x}$ from 
\begin{align}
	\myvec{\frac{10}{8} & \frac{3}{8} \\ \frac{3}{8} & \frac{25}{16} }
	\vec{x} = \myvec{-3 \\ 2} 
\end{align}
Solving the augmented matrix, we get
\begin{align}
     \myvec{\frac{10}{8} & \frac{3}{8} & -3 \\ \frac{3}{8} & \frac{25}{16} & 2}
	\xleftrightarrow[]{R_1 \leftarrow -\frac{3}{10}R_1}
     \myvec{1 & \frac{3}{10} &-\frac{24}{10} \\ \frac{3}{8} & \frac{25}{16} & 2}
	\\
	\xleftrightarrow[]{R_2 \leftarrow R_2-\frac{3}{8}R_1}
	\myvec{1 & \frac{3}{10} &-\frac{24}{10} \\ 
		0 & \frac{29}{20} & \frac{58}{20} }
	\rightarrow \myvec{1 & \frac{3}{10} &-\frac{24}{10} \\ 0 & 1 & 2 } \\
	\xleftrightarrow[]{R_1 \leftarrow R_1-\frac{3}{10}R_2}
	\myvec{1 & 0 & -3 \\ 0 & 1 & 2}
\end{align}
Hence, the solution is given by
\begin{align}
	\vec{x} = \myvec{ -3 \\ 2 } 	\label{xverifyeq}
\end{align}
Comparing the results in Eq.(\ref{xeq}) and (\ref{xverifyeq}), it is
concluded that the solution is verified.

\end{document}
